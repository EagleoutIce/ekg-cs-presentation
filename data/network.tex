% old network symbol code from me
\colorlet{switch}{gray!95!btdl@color@secondary}
\makeatletter
\pgfkeys{/pgf/.cd,
  parallelepiped offset x/.initial=2mm, parallelepiped offset y/.initial=2mm
}
\pgfdeclareshape{parallelepiped}{
\inheritsavedanchors[from=rectangle] % this is nearly a rectangle
\inheritanchorborder[from=rectangle] \inheritanchor[from=rectangle]{north} \inheritanchor[from=rectangle]{north west}
\inheritanchor[from=rectangle]{north east} \inheritanchor[from=rectangle]{center} \inheritanchor[from=rectangle]{west}
\inheritanchor[from=rectangle]{east} \inheritanchor[from=rectangle]{mid} \inheritanchor[from=rectangle]{mid west}
\inheritanchor[from=rectangle]{mid east} \inheritanchor[from=rectangle]{base} \inheritanchor[from=rectangle]{base west}
\inheritanchor[from=rectangle]{base east} \inheritanchor[from=rectangle]{south} \inheritanchor[from=rectangle]{south west}
\inheritanchor[from=rectangle]{south east}
\backgroundpath{
% store lower right in xa/ya and upper right in xb/yb
\southwest \pgf@xa=\pgf@x \pgf@ya=\pgf@y
\northeast \pgf@xb=\pgf@x \pgf@yb=\pgf@y
\pgfmathsetlength\pgfutil@tempdima{\pgfkeysvalueof{/pgf/parallelepiped
  offset x}}
\pgfmathsetlength\pgfutil@tempdimb{\pgfkeysvalueof{/pgf/parallelepiped
  offset y}}
\def\ppd@offset{\pgfpoint{\pgfutil@tempdima}{\pgfutil@tempdimb}}
\pgfpathmoveto{\pgfqpoint{\pgf@xa}{\pgf@ya}}
\pgfpathlineto{\pgfqpoint{\pgf@xb}{\pgf@ya}}
\pgfpathlineto{\pgfqpoint{\pgf@xb}{\pgf@yb}}
\pgfpathlineto{\pgfqpoint{\pgf@xa}{\pgf@yb}}
\pgfpathclose
\pgfpathmoveto{\pgfqpoint{\pgf@xb}{\pgf@ya}}
\pgfpathlineto{\pgfpointadd{\pgfpoint{\pgf@xb}{\pgf@ya}}{\ppd@offset}}
\pgfpathlineto{\pgfpointadd{\pgfpoint{\pgf@xb}{\pgf@yb}}{\ppd@offset}}
\pgfpathlineto{\pgfpointadd{\pgfpoint{\pgf@xa}{\pgf@yb}}{\ppd@offset}}
\pgfpathlineto{\pgfqpoint{\pgf@xa}{\pgf@yb}}
\pgfpathmoveto{\pgfqpoint{\pgf@xb}{\pgf@yb}}
\pgfpathlineto{\pgfpointadd{\pgfpoint{\pgf@xb}{\pgf@yb}}{\ppd@offset}}
}}

\tikzset{
l3 switch/.style={
  parallelepiped,fill=switch, draw=white,
  minimum width=.75cm,
  minimum height=.75cm,
  parallelepiped offset x=1.75mm,
  parallelepiped offset y=1.25mm,
  path picture={
    \node[fill=white,
      circle,
      minimum size=6pt,
      inner sep=0pt,
      append after command={
        \pgfextra{
          \foreach \angle in {0,45,...,360}
          \draw[-latex,fill=white] (\tikzlastnode.\angle)--++(\angle:2.25mm);
        }
      }
    ]
      at ([xshift=-.75mm,yshift=-.5mm]path picture bounding box.center){};
  }
},
ports/.style={
  line width=.15pt,
  top color=gray!20,
  bottom color=gray!80
},
rack switch/.style={
  parallelepiped,fill=white, draw,
  minimum width=1.25cm,
  minimum height=.25cm,
  parallelepiped offset x=2mm,
  parallelepiped offset y=1.25mm,
  xscale=-1,
  line join=round,
  path picture={
    \draw[top color=gray!5,bottom color=gray!40]
    (path picture bounding box.south west) rectangle
    (path picture bounding box.north east);
    \coordinate (A-west) at ([xshift=-.2cm]path picture bounding box.west);
    \coordinate (A-center) at ($(path picture bounding box.center)!0!(path
      picture bounding box.south)$);
    \foreach \x in {.275,.525,.775}{
      \draw[ports]([yshift=-.05cm]$(A-west)!\x!(A-center)$)
        rectangle +(.1,.05);
      \draw[ports]([yshift=-.125cm]$(A-west)!\x!(A-center)$)
        rectangle +(.1,.05);
      }
    \coordinate (A-east) at (path picture bounding box.east);
    \foreach \x in {.085,.21,.335,.455,.635,.755,.875,1}{
      \draw[ports]([yshift=-.1125cm]$(A-east)!\x!(A-center)$)
        rectangle +(.05,.1);
    }
  }
},
server/.style={
  parallelepiped,
  fill=white, draw,
  minimum width=.35cm,
  minimum height=.75cm,
  parallelepiped offset x=3mm,
  parallelepiped offset y=2mm,
  line join=round,
  xscale=-1,
  path picture={
    \draw[top color=gray!5,bottom color=gray!40]
    (path picture bounding box.south west) rectangle
    (path picture bounding box.north east);
    \coordinate (A-center) at ($(path picture bounding box.center)!0!(path
      picture bounding box.south)$);
    \coordinate (A-west) at ([xshift=-.575cm]path picture bounding box.west);
    \draw[ports]([yshift=.1cm]$(A-west)!0!(A-center)$)
      rectangle +(.2,.065);
    \draw[ports]([yshift=.01cm]$(A-west)!.085!(A-center)$)
      rectangle +(.15,.05);
    \fill[black]([yshift=-.35cm]$(A-west)!-.1!(A-center)$)
      rectangle +(.235,.0175);
    \fill[black]([yshift=-.385cm]$(A-west)!-.1!(A-center)$)
      rectangle +(.235,.0175);
    \fill[black]([yshift=-.42cm]$(A-west)!-.1!(A-center)$)
      rectangle +(.235,.0175);
  }
},
interface/.style={draw, rectangle, rounded corners, font=\LARGE\sffamily},
ethernet/.style={interface, fill=yellow!50},% ethernet interface
serial/.style={interface, fill=green!70},% serial interface
speed/.style={sloped, anchor=south, font=\large\sffamily},% l ine speed at edge
route/.style={draw, shape=single arrow, single arrow head extend=4mm,
  minimum height=1.7cm, minimum width=3mm, white, fill=switch!20,
  drop shadow={opacity=.8, fill=switch}, font=\tiny}% inroute/outroute arrows
}

\def\shift{1.3cm}
% The router icon
\newcommand*\router[1]{
\begin{tikzpicture}
  \coordinate (ll) at (-3,1);
  \coordinate (lr) at (3,1);
  \coordinate (ul) at (-3,2);
  \coordinate (ur) at (3,2);
  \shade [shading angle=90, left color=switch, right color=switch!10!white] (ll)
    arc (-180:-60:3cm and .75cm) -- +(0,1) arc (-60:-180:3cm and .75cm)
    -- cycle;
  \shade [shading angle=270, right color=switch, left color=switch!10!white] (lr)
    arc (0:-60:3cm and .75cm) -- +(0,1) arc (-60:0:3cm and .75cm) -- cycle;
  \draw [thick] (ll) arc (-180:0:3cm and .75cm)
    -- (ur) arc (0:-180:3cm and .75cm) -- cycle;
  \draw [thick, shade, upper left=switch, lower left=switch,
    upper right=switch, lower right=switch!10!white] (ul)
    arc (-180:180:3cm and .75cm);
  \node at (0,.75){\color{switch!10!black}\Huge #1};% The name of the router
  %  We will now draw the symbol of the course
  \draw[ultra thick,darkgray] (-1.9,1.75) -- ++ (1.25,0) -- ++(1.5,.6) -- ++(1.25,0);
  \draw[ultra thick,darkgray] (-1.9,2.35) -- ++ (1.25,0) -- ++(1.5,-.6) -- ++(1.25,0);
\end{tikzpicture}}
