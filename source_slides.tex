\errorcontextlines99999
\documentclass[aspectratio=169,usepdftitle=true]{beamer}

\usepackage[T1]{fontenc}
\usepackage[utf8]{inputenc}
\usepackage[english]{babel}
\usepackage{microtype}

\usepackage{etoolbox}

\makeatletter
\appto\input@path{{lib/color-palettes/}{lib/sopra-collection/sopra-listings}{lib/tikzpingus/tex/}{lib/fancyqr/}}

\usepackage{lib/color-palettes/color-palettes}
\usepackage[cpalette,encoding,defaultfont,fakeminted]{lib/sopra-collection/sopra-listings/sopra-listings}

\solLoadLanguage{R}
\solsetmintedstyle{plain number}

\usepackage{lib/beamer-themes/dividing-lines/beamerthemedividing-lines}
\SetColorProfile{42, 42, 42}{249, 166, 2}{137,64,75}
\definecolor{btdl@color@background}{rgb}{252, 252, 252}

\usepackage[backend=bibtex8,style=numeric]{biblatex}
\usepackage{csquotes}
\let\say\enquote
% \addbibresource{references.bib}
% \outro{\printBibCommand}

\usepackage{tikz}
\usetikzlibrary{arrows.meta,backgrounds,shapes.symbols,decorations.pathreplacing}
\usepackage{forest}
\usepackage{lib/tikzpingus/tex/tikzpingus}

% too short of a presentation
\sectionbannerfalse

\def\nl{\texorpdfstring{\\}{\space}}
\def\nlq{\texorpdfstring{\\\strut\quad}{\space}}
\def\nlqq{\texorpdfstring{\\\strut\qquad}{\space}}
\def\Off#1{\texorpdfstring{\textit{\textcolor{gray}{#1}}}{#1}}
\robustify\Off
\title{\Off{Von}\nlq Physik,\nlq Informatik\nlqq\Off{und dem ganzen Rest}}

\date{\today}

\author[F.~Sihler]{Florian Sihler}
\email{florian.sihler@uni-ulm.de}
\license{\ccbysa}
\usepackage{fontawesome}

\def\PlaceFramenumberUpperLeft{}
\def\TitlepageSlideAlign{c}
\def\titlepage{%
\begin{tikzpicture}[overlay,remember picture]
   \node[align=left,font=\usebeamerfont{title}\usebeamercolor{title}] at (current page.center) {%
      \MakeUppercase{\inserttitle}\endgraf
   };
   \node[below left] at(current page.north east) {\color{gray}\href{https://github.com/EagleoutIce/ekg-cs-presentation}{\faGithub}};
\end{tikzpicture}
}

\tikzset{
   T/.style={text=gray,font=\scriptsize\sffamily}
}
\errorcontextlines9999
\lstset{add to literate=
   {<-}{{{\!\(\leftarrow\)\!}}}2
}

\begin{document}

% TODO: remove footer compleely
% \section{Cryptography}
\begin{frame}[c]{~}
   \begin{tikzpicture}[overlay,remember picture]
      \onslide<2->{\node (@) at(current page.center) {\usebeamerfont{title}204.715};}
      % TODO: join fonts
      \onslide<3->{\node[below] at(@.south) {\color{gray}\sbfamily19.973};}
      \onslide<2->{\node[above right,T] at(current page.south west) {\href{https://web.archive.org/web/20230607125232/https://www.cvedetails.com/}{https://www.cvedetails.com/}};}
   \end{tikzpicture}
\end{frame}

\newsavebox\pinguA
\newsavebox\pinguB
\newsavebox\pinguE

\savebox\pinguA{\tikz{\pingu[body=btdl@color@primary, body type=legacy, right wing wave, conical hat, eyes wink,heart]}}
\savebox\pinguB{\tikz{\pingu[body=btdl@color@primary,eyes shiny,feet sit, wings grab, cup=pingu@purple!40!btdl@color@primary]}}

\savebox\pinguE{\tikz{\pingu[body=btdl@color@primary,eyes angry, sunglasses, cloak=black!92!gray, cloak cap=!hide, devil horns=btdl@color@primary,tie=black!92!gray, eye patch right]}}

\begin{frame}[c]{}
\begin{tikzpicture}[
   overlay,remember picture,
   every path/.append style={line cap=round}
]
   \coordinate (c) at (current page.center);

   \onslide<2->{\node[scale=.75] (a) at ([xshift=-4.5cm,yshift=1cm]c) {\usebox\pinguA};
   \node[below] (ad) at(a.south) {Alice};}

   \onslide<3->{\node[scale=.75] (b) at ([xshift=4.5cm,yshift=1cm]c) {\usebox\pinguB};
   \node (bd) at(b.south|-ad.east) {Bob};}

   \onslide<4->{
      \draw[-Kite,gray, dashed] (a) -- (b) node[midway,above,T,text=btdl@color@primary,font=\sbfamily\large] {Hallo} coordinate[pos=.5] (m);
   }

   \onslide<5->{
      \node[scale=.75] (e) at ([yshift=-2cm]c) {\usebox\pinguE};
      \node[below] at(e.south) {Eve};
   }

   \onslide<6->{
      \fill (m) circle[radius=2pt];
      \draw (e.70) to[out=85,in=300] (m);
   }
\end{tikzpicture}
\end{frame}

\newsavebox\pinguAGray
\newsavebox\pinguBGray

\def\MixinOnGray{28!btdl@color@background}

\savebox\pinguAGray{\tikz{\pingu[body=btdl@color@primary, body type=legacy, right wing wave, conical hat, eyes wink,heart,:mix-all=\MixinOnGray]}}
\savebox\pinguBGray{\tikz{\pingu[body=btdl@color@primary,eyes shiny,feet sit, wings grab, cup=pingu@purple!40!btdl@color@primary,:mix-all=\MixinOnGray]}}

\colorlet{@highlight}{btdl@color@primary}
\newcommand<>\HighlightOn{%
   \only#1{\colorlet{black}{black!\MixinOnGray}\colorlet{gray}{gray!\MixinOnGray}\color{black}}%
}
\begin{frame}[c]{}
\begin{tikzpicture}[
   overlay,remember picture,
   every path/.append style={line cap=round},
   font=\sbfamily\large
]
   \def\ShiftUp{0cm}
   \only<8->{\def\ShiftUp{2.25cm}}
   \coordinate[yshift=\ShiftUp-1.25mm] (c) at (current page.center);
   \node[scale=.75] (a) at ([xshift=-4.5cm]c) {\only<-5|handout:0>{\usebox\pinguA}\only<6->{\usebox\pinguAGray}};
   \node[below,font=\sffamily] (ad) at(a.south) {\HighlightOn<6->Alice};

   \node[scale=.75] (b) at ([xshift=4.5cm]c) {\only<-5|handout:0>{\usebox\pinguB}\only<6->{\usebox\pinguBGray}};
   \node[font=\sffamily] (bd) at(b.south|-ad.east) {\HighlightOn<6->Bob};

   \onslide<2->{
      {\HighlightOn<6->
      \draw[-Kite,gray, dashed] (a) -- (b) node[midway,above,T,text=btdl@color@primary] {} coordinate[pos=.5] (m);}

      \node[above right,yshift=-1.5mm,xshift=-2.5cm] (encrypt) at(a.north east) {%
         \HighlightOn<6->%
         \only<2|handout:0>{\phantom{encrypt(}Hallo}%
         \only<3->{encrypt(\textcolor{gray}{Hallo}, \setbox0=\hbox{\textcolor{gray!\MixinOnGray}{\faKey}}\kern.5\wd0\clap{\copy0}\clap{\only<6->{\color{@highlight}}7}\kern.5\wd0)}%
      };
   }
   \onslide<4->{
      \node[above] at (m.north) {\HighlightOn<6->Ohssv};% \faLongArrowRight\space
   }

   \onslide<5->{
      \node[above left,yshift=-1.5mm,xshift=.5cm] (decrypt) at(b.north east|-a.north east) {%
         \HighlightOn<6->%
         decrypt(\textcolor{gray}{Ohssv}, \setbox0=\hbox{\textcolor{gray!\MixinOnGray}{\faKey}}\kern.5\wd0\clap{\copy0}\clap{\only<6->{\color{@highlight}}7}\kern.5\wd0)%
      };
   }
   \onslide<7->{
      \node[yshift=-.66cm] at (m.south) {Symmetrisch};
   }

   \begin{onlyenv}<9->
   \coordinate[yshift=-\ShiftUp-1.25mm] (c2) at (current page.center);
   \node[scale=.75] (a) at ([xshift=-4.5cm]c2) {\usebox\pinguAGray};
   \node[below,font=\sffamily] (ad) at(a.south) {\HighlightOn<6->Alice};

   \node[scale=.75] (b) at ([xshift=4.5cm]c2) {\usebox\pinguBGray};
   \node[font=\sffamily] (bd) at(b.south|-ad.east) {\HighlightOn<6->Bob};

      {\HighlightOn<6->
      \draw[-Kite,gray, dashed] (a) -- (b) node[midway,above,T,text=btdl@color@primary] {} coordinate[pos=.5] (m);}

      \node[above right,yshift=-1.5mm,xshift=-2.5cm] (encrypt) at(a.north east) {%
         \HighlightOn<6->encrypt(\textcolor{gray}{Hallo}, \setbox0=\hbox{\textcolor{gray!\MixinOnGray}{\faKey}}\kern.5\wd0\clap{\copy0}\clap{\only<6->{\color{@highlight}}7}\kern.5\wd0)%
      };
      \node[above] at (m.north) {\HighlightOn<6->Ohssv};% \faLongArrowRight\space

      \node[above left,yshift=-1.5mm,xshift=.5cm] (decrypt) at(b.north east|-a.north east) {%
         \HighlightOn<6->%
         decrypt(\textcolor{gray}{Ohssv}, \setbox0=\hbox{\textcolor{gray!\MixinOnGray}{\faKey}}\kern.5\wd0\clap{\copy0}\clap{\only<6->{\color{@highlight}}4}\kern.5\wd0)%
      };
      \node[yshift=-.66cm] at (m.south) {Asymmetrisch};
   \end{onlyenv}

\end{tikzpicture}
\end{frame}


% TODO: viele variationen und andere Optionen wie das RSA Verfahren
\begin{frame}[c]{}
% Alice wählt zwei zufällige unabhängige Primzahlen ungefähr der gleichen Größenordnung:
\begin{tikzpicture}[overlay,remember picture]
\begin{onlyenv}<2-5|handout:1>
\node[text width=.9\paperwidth] at(current page.center) {%
\begin{center}
   \onslide<2->{\(\mathbf{p} = 7\) \qquad \(\mathbf{g} = 3\)}\vspace*{1.5em}
\end{center}
\begin{center}
   \onslide<3->{\(\mathbf{N} = p \cdot q = 21\)}\\
   \onslide<3->{{\normalsize\textcolor{gray}{RSA-Modul}}}
\end{center}
\begin{center}
   \onslide<4->{\(\boldsymbol\varphi\mathbf{(N)} = (p - 1) \cdot (q - 1) = 12\)}\\
   \onslide<4->{{\normalsize\textcolor{gray}{Eulersche \(\varphi\)-Funktion}}}\vspace*{1.5em}
\end{center}
\begin{center}
   \onslide<5->{\(\mathbf{e} = 5\) \qquad \(\mathbf{d} = 17\)}\\
   \onslide<5->{{\normalsize\textcolor{gray}{\textbf{e}ncrypt und \textbf{d}ecrypt}}}
   % teilerfremd zu \(\varphi(N)\), und 1 < e < \(\varphi(N) - 1\)}\\
\end{center}
};
\end{onlyenv}

\begin{onlyenv}<6-|handout:2>
   \node[below left=5mm] at(current page.north east) {\(N = 21\), \(e = 5\), \(d = 17\)};
\node[text width=.9\paperwidth] at(current page.center) {%
\begin{center}
   \onslide<7->{\(\mathbf{m} = 4\)}\\
   \onslide<7->{{\normalsize\textcolor{gray}{Nachricht}}}\vspace*{1.5em}
\end{center}
\begin{center}
   \onslide<8->{\(\mathbf{c} = m^e \bmod N = 4^5 \bmod 21 = 16\)}\\
   \onslide<8->{{\normalsize\textcolor{gray}{Verschlüsselte Nachricht}}}\vspace*{1em}
\end{center}
\begin{center}
   \onslide<9->{\(\mathbf{m} = c^d \bmod N = 16^{17} \bmod 21 = 4\)}\\
   \onslide<9->{{\normalsize\textcolor{gray}{Entschlüsselte Nachricht}}}
\end{center}
};
\end{onlyenv}
% von e& N  auf D zu schließen ist *sehr schwer* das "Faktorisierungsproblem"
\end{tikzpicture}
% Berechnen des Moduls \(N = p \cdot q\) und Eulersche Phi-Funktion \(\varphi(N) = (p-1) \cdot (q-1)\)
% whle zu phi(N) teilerfremde Zahl e mit 1 < e < phi(N)
% berechne d mit e*d = 1 mod phi(N)
% \begin{tikzpicture}[overlay, remember picture]
%    \coordinate (c) at (current page.center);
%    \node[scale=.66] (a) at ([xshift=-5cm]c) {\usebox\pinguA};
%    \node[scale=.66] (b) at ([xshift=5cm]c) {\usebox\pinguB};

%    \draw[line cap=round, thick]  ([yshift=3cm,xshift=8mm]a.east) -- ++(0,-6cm);
%    \draw[line cap=round, thick]  ([yshift=3cm,xshift=-8mm]b.west) -- ++(0,-6cm);

%    \node[below=5mm] at(current page.north) {\(p = 7\) \qquad \(g = 3\)};
% \end{tikzpicture}
\end{frame}


\newsavebox\pinguShock
\savebox\pinguShock{\tikz{\pingu[wings shock, eyes shock,heart=gray!30]}}
\begin{frame}[c]{}
\begin{tikzpicture}[overlay,remember picture]
   \onslide<2->{\node at (current page.center) {\usebeamerfont{title}Faktorisierungsproblem};}
   \onslide<3->{\node[above left=-1.25cm] at(current page.south east) {\rotatebox{30}{\usebox\pinguShock}};}
\end{tikzpicture}
\end{frame}

% Überleitung Quantum Computing
\begin{frame}[c]{}
   \begin{tikzpicture}[overlay,remember picture]
      \onslide<2->{\node at (current page.center) {\usebeamerfont{title}0\quad\textcolor{lightgray}{/}\quad 1};}
   \end{tikzpicture}
   % TODO: qbits
\end{frame}


\end{document}