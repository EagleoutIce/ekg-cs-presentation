
% TODO: remove footer compleely
% \section{Cryptography}
\begin{frame}[c]{~}
   \begin{tikzpicture}[@O]
      \onslide<2->{\node (@) at(current page.center) {\usebeamerfont{title}205.770};}
      % TODO: join fonts
      \onslide<3->{\node[below] at(@.south) {\color{gray}\sbfamily19.973};}
      \onslide<2->{\node[above right,T] at(current page.south west) {\href{https://web.archive.org/web/20230620131851/https://www.cvedetails.com/}{https://www.cvedetails.com/}};}
   \end{tikzpicture}
   \note[itemize]{%
      \item 205.770 Schwachstellen am Dienstag den 20.06.2023
      \item 19.973 Schwachstellen der höchsten Kritikalität
   }
\end{frame}


\newsavebox\FinalCryptIntroSlide
\begin{frame}[c]{}
\hypertarget{@FinalCrypt}{}%
\begin{lrbox}\FinalCryptIntroSlide
\begin{tikzpicture}[every path/.append style={line cap=round}]
   \coordinate (c) at (0,0);

   \onslide<2->{\node[scale=.75] (a) at ([xshift=-4.5cm,yshift=1cm]c) {\usebox\pinguA};
   \node[below] (ad) at(a.south) {Alice};}

   \onslide<3->{
      \node[scale=.75] (b) at ([xshift=4.5cm,yshift=1cm]c) {\usebox\pinguB};
      \node (bd) at(b.south|-ad.east) {Bob};
      \draw[-Kite,gray, dashed] (a) -- (b) node[midway,above,T,text=btdl@color@primary,font=\sbfamily\large] {Hallo} coordinate[pos=.5] (m);
   }

   \onslide<4->{
      \node[scale=.75] (e) at ([yshift=-2cm]c) {\usebox\pinguE};
      \node[below] at(e.south) {Eve};
      \fill (m) circle[radius=2pt];
      \draw (e.70) to[out=85,in=300] (m);
   }
\end{tikzpicture}%
\end{lrbox}
\begin{tikzpicture}[@O]
   \node at (current page.center) {\usebox\FinalCryptIntroSlide};
\end{tikzpicture}
\global\setbox\FinalCryptIntroSlide=\hbox{\usebox\FinalCryptIntroSlide}%
\end{frame}

\colorlet{@highlight}{btdl@color@primary}
\newcommand<>\HighlightOn{%
   \only#1{\colorlet{black}{black!\MixinOnGray}\colorlet{gray}{gray!\MixinOnGray}\color{black}}%
}
\begin{frame}[c]{}
\begin{tikzpicture}[
   @O,
   every path/.append style={line cap=round},
   font=\sbfamily\large
]
   \def\ShiftUp{0cm}
   \only<8->{\def\ShiftUp{2.25cm}}
   \coordinate[yshift=\ShiftUp-1.25mm] (c) at (current page.center);
   \node[scale=.75] (a) at ([xshift=-4.5cm]c) {\only<-5|handout:0>{\usebox\pinguA}\only<6->{\usebox\pinguAGray}};
   \node[below,font=\sffamily] (ad) at(a.south) {\HighlightOn<6->Alice};

   \node[scale=.75] (b) at ([xshift=4.5cm]c) {\only<-5|handout:0>{\usebox\pinguB}\only<6->{\usebox\pinguBGray}};
   \node[font=\sffamily] (bd) at(b.south|-ad.east) {\HighlightOn<6->Bob};

   \onslide<2->{
      {\HighlightOn<6->
      \draw[-Kite,gray, dashed] (a) -- (b) node[midway,above,T,text=btdl@color@primary] {} coordinate[pos=.5] (m);}

      \node[above right,yshift=-1.5mm,xshift=-2.5cm] (encrypt) at(a.north east) {%
         \HighlightOn<6->%
         \only<2|handout:0>{\phantom{encrypt(}Hallo}%
         \only<3->{encrypt(\textcolor{gray}{Hallo}, \setbox0=\hbox{\textcolor{gray!\MixinOnGray}{\faKey}}\kern.5\wd0\clap{\copy0}\clap{\only<6->{\color{@highlight}}7}\kern.5\wd0)}%
      };
   }
   \onslide<4->{
      \node[above] at (m.north) {\HighlightOn<6->Ohssv};% \faLongArrowRight\space
   }

   \onslide<5->{
      \node[above left,yshift=-1.5mm,xshift=.5cm] (decrypt) at(b.north east|-a.north east) {%
         \HighlightOn<6->%
         decrypt(\textcolor{gray}{Ohssv}, \setbox0=\hbox{\textcolor{gray!\MixinOnGray}{\faKey}}\kern.5\wd0\clap{\copy0}\clap{\only<6->{\color{@highlight}}7}\kern.5\wd0)%
      };
   }
   \onslide<7->{
      \node[yshift=-.66cm] at (m.south) {Symmetrisch};
   }

   \begin{onlyenv}<9->
   \coordinate[yshift=-\ShiftUp-1.25mm] (c2) at (current page.center);
   \node[scale=.75] (a) at ([xshift=-4.5cm]c2) {\usebox\pinguAGray};
   \node[below,font=\sffamily] (ad) at(a.south) {\HighlightOn<6->Alice};

   \node[scale=.75] (b) at ([xshift=4.5cm]c2) {\usebox\pinguBGray};
   \node[font=\sffamily] (bd) at(b.south|-ad.east) {\HighlightOn<6->Bob};

      {\HighlightOn<6->
      \draw[-Kite,gray, dashed] (a) -- (b) node[midway,above,T,text=btdl@color@primary] {} coordinate[pos=.5] (m);}

      \node[above right,yshift=-1.5mm,xshift=-2.5cm] (encrypt) at(a.north east) {%
         \HighlightOn<6->encrypt(\textcolor{gray}{Hallo}, \setbox0=\hbox{\textcolor{gray!\MixinOnGray}{\faKey}}\kern.5\wd0\clap{\copy0}\clap{\only<6->{\color{@highlight}}7}\kern.5\wd0)%
      };
      \node[above] at (m.north) {\HighlightOn<6->Ohssv};% \faLongArrowRight\space

      \node[above left,yshift=-1.5mm,xshift=.5cm] (decrypt) at(b.north east|-a.north east) {%
         \HighlightOn<6->%
         decrypt(\textcolor{gray}{Ohssv}, \setbox0=\hbox{\textcolor{gray!\MixinOnGray}{\faKey}}\kern.5\wd0\clap{\copy0}\clap{\only<6->{\color{@highlight}}4}\kern.5\wd0)%
      };
      \node[yshift=-.66cm] at (m.south) {Asymmetrisch};
   \end{onlyenv}

\end{tikzpicture}
\end{frame}


\begin{frame}{}
\begin{tikzpicture}[@O]
   \onslide<2->{
      \node[right=5mm,scale=.6,yshift=1cm] (a) at(current page.west) {\usebox\pinguA};
      \node[left=5mm,scale=.6,yshift=1cm] (b) at(current page.east) {\usebox\pinguB};
      \coordinate[xshift=2.85cm] (@a) at (a.east);
      \coordinate[xshift=-2.85cm] (@b) at (b.west);
      % \draw[gray,densely dashed] (current page.north-|@a) -- (current page.south-|@a);
      % \draw[gray,densely dashed] (current page.north-|@b) -- (current page.south-|@b);
   }
   \scope[every node/.append style={font=\sbfamily\small}]
   \onslide<3->{
      \node[below=5mm] at(current page.north) {\large\(p = 11\)\qquad\(g = 7\)};
   }
   \onslide<4->{
      \node[above] at(a.north) {\(a = ~~\clap{\text{\usebox\LightRandom}}\clap{\sbfamily 3}~~\)};
   }
   \onslide<5->{
      \tikzset{@/.style={}}
      \only<6->{\tikzset{@/.style={color=gray}}}
      \node[above right,xshift=5mm,@] (a-base) at(a.east) {\(g^{\mathit{a}} \bmod p\)};
   }
   \onslide<6->{
      \node[below right] (ta) at (a-base.south west) {\(7^{\mathit{3}} \bmod 11 = \textbf{2}\)};
   }
   \onslide<7->{
      \node[above] at(b.north) {\sbfamily\large \(b = ~~\clap{\text{\usebox\LightRandom}}\clap{\sbfamily2}~~\)};
   }
   \onslide<8->{
      \node[above left,xshift=-5mm,gray] (b-base) at(b.west) {\(g^{\mathit{a}} \bmod p\)};
      \node[below left] (tb) at (b-base.south east) {\(7^{\mathit{2}} \bmod 11 = \textbf{5}\)};
   }
   \onslide<9->{
      \draw[-Kite,gray] ([xshift=3mm]ta.east) to[edge node={node[above,sloped,scale=.75,pos=.2] {\(g^a \bmod p = 2\)}}] ([yshift=-1.25cm]ta-|tb.west) coordinate (ca);
      \draw[-Kite,gray] ([xshift=-3mm]tb.west) to[edge node={node[above,sloped,scale=.75,pos=.2] {\(g^b \bmod p = 5\)}}] ([yshift=-1.25cm]tb-|ta.east) coordinate (cb);
   }
   \onslide<10->{
      \node[right=3mm] at(ca) {\(5^3 \bmod 11 = \textbf{4}\)};
   }
   \onslide<11->{
      \node[left=3mm] at(cb) {\(2^2 \bmod 11 = \textbf{4}\)};
   }
   \endscope

   \onslide<12->{
      \node[above=8mm] at(current page.south) {\usebeamerfont{subtitle}Diffie-Hellman Schlüsselaustausch};
   }
\end{tikzpicture}
\note[itemize]{
   \item {(g^a mod p)^b mod p = (g^a)^b mod p = g^(a*b) mod p = g^(b * a) mod p = (g^b mod p)^a mod p}
   \item {Estimate: 2048 bit primes sind 10^9 mal schwerer zu knacken als 1024 bit primes}
   \item Whitfield Diffie, Martin Hellman und Ralph Merkle
}(4)
\end{frame}

% TODO: viele variationen und andere Optionen wie das RSA Verfahren
% rsa backup
\AtEndDocument{%
% Mathematik interessiert uns erstmal gar nicht so
\begin{frame}[c]{}
% Alice wählt zwei zufällige unabhängige Primzahlen ungefähr der gleichen Größenordnung:
\begin{tikzpicture}[@O]
\begin{onlyenv}<2-5|handout:1>
\node[text width=.9\paperwidth] at(current page.center) {%
\begin{center}
   \onslide<2->{\(\mathbf{p} = 7\) \qquad \(\mathbf{q} = 3\)}\vspace*{1.5em}
\end{center}
\begin{center}
   \onslide<3->{\(\mathbf{N} = p \cdot q = 21\)}\\
   \onslide<3->{{\normalsize\textcolor{gray}{RSA-Modul}}}
\end{center}
\begin{center}
   \onslide<4->{\(\boldsymbol\Phi\mathbf{(N)} = (p - 1) \cdot (q - 1) = 12\)}\\
   \onslide<4->{{\normalsize\textcolor{gray}{Eulersche \(\Phi\)-Funktion}}}\vspace*{1.5em}
\end{center}
\begin{center}
   \onslide<5->{\(\mathbf{e} = 5\) \qquad \(\mathbf{d} = 17\)}\\
   \onslide<5->{{\normalsize\textcolor{gray}{\textbf{e}ncrypt und \textbf{d}ecrypt}}}
   % teilerfremd zu \(\varphi(N)\), und 1 < e < \(\varphi(N) - 1\)}\\
\end{center}
};
\end{onlyenv}

\begin{onlyenv}<6-|handout:2>
   \node[below left=5mm] at(current page.north east) {\(N = 21\), \(e = 5\), \(d = 17\)};
\node[text width=.9\paperwidth] at(current page.center) {%
\begin{center}
   \onslide<7->{\(\mathbf{m} = 4\)}\\
   \onslide<7->{{\normalsize\textcolor{gray}{Nachricht}}}\vspace*{1.5em}
\end{center}
\begin{center}
   \onslide<8->{\(\mathbf{c} = m^e \bmod N = 4^5 \bmod 21 = 16\)}\\
   \onslide<8->{{\normalsize\textcolor{gray}{Verschlüsselte Nachricht}}}\vspace*{1em}
\end{center}
\begin{center}
   \onslide<9->{\(\mathbf{m} = c^d \bmod N = 16^{17} \bmod 21 = 4\)}\\
   \onslide<9->{{\normalsize\textcolor{gray}{Entschlüsselte Nachricht}}}
\end{center}
};
\end{onlyenv}
% von e& N  auf D zu schließen ist *sehr schwer* das "Faktorisierungsproblem"
\end{tikzpicture}
% Berechnen des Moduls \(N = p \cdot q\) und Eulersche Phi-Funktion \(\varphi(N) = (p-1) \cdot (q-1)\)
% whle zu phi(N) teilerfremde Zahl e mit 1 < e < phi(N)
% berechne d mit e*d = 1 mod phi(N)
% \begin{tikzpicture}[overlay,remember picture]
%    \coordinate (c) at (current page.center);
%    \node[scale=.66] (a) at ([xshift=-5cm]c) {\usebox\pinguA};
%    \node[scale=.66] (b) at ([xshift=5cm]c) {\usebox\pinguB};

%    \draw[line cap=round, thick]  ([yshift=3cm,xshift=8mm]a.east) -- ++(0,-6cm);
%    \draw[line cap=round, thick]  ([yshift=3cm,xshift=-8mm]b.west) -- ++(0,-6cm);

%    \node[below=5mm] at(current page.north) {\(p = 7\) \qquad \(g = 3\)};
% \end{tikzpicture}
\end{frame}
}


\newsavebox\pinguShock
\savebox\pinguShock{\tikz{\pingu[wings shock, eyes shock,heart=gray!30]}}
\begin{frame}[c]{}
\begin{tikzpicture}[@O]
   \onslide<2->{\node at (current page.center) {\usebeamerfont{title}Faktorisierungsproblem};}
   \onslide<3->{\node[above left=-1.25cm] at(current page.south east) {\rotatebox{30}{\usebox\pinguShock}};}
\end{tikzpicture}
\end{frame}

% Überleitung Quantum Computing
\def\q#1{\ensuremath{\raisebox{1.5pt}{$\boldsymbol|$}\text{\sbfamily#1}\raisebox{1.5pt}{$\boldsymbol\rangle$}}}
\begin{frame}[c]{}
   \begin{tikzpicture}[@O]
      \def\MoveUp{0cm}
      \only<3->{\def\MoveUp{.25\paperheight}}
      \onslide<2->{\node at ([yshift=\MoveUp]current page.center) {\usebeamerfont{title}0\quad\textcolor{lightgray!80}{/}\quad 1};}

      \onslide<3->{
         \node[below right=2mm] at(current page.north west) {\sbfamily Bit};
         \draw[lightgray] (current page.east)--(current page.west);
      }

      % \raisebox{-3pt}{\clap{\(c_0 \cdot \q0 + c_1 \cdot \q1\)}}
      \onslide<4->{
         \node[below right=2mm] at(current page.west) {\sbfamily Qubit};
         \node at ([yshift=-\MoveUp]current page.center) {\usebeamerfont{title}\q0\quad\textcolor{lightgray!80}{/}\only<5->{\quad\q{{\(\underset{\clap{\color{gray}\footnotesize Superposition}}{\Psi}\)}}\quad\textcolor{lightgray!80}{/}}\quad \q1};
      }
   \end{tikzpicture}
\end{frame}

% bisher sind Qubits aber noch nicht besser, weil wir bei der Messung immer nur einen der beiden Zustände (0/1) erhalten

\begin{frame}[c]{}
\begin{tikzpicture}[@O]
   \onslide<2->{
      \node[align=center,text width=.9\paperwidth] at(current page.center) {%
\begin{align*}
   \q{$\Psi$} &= a \cdot \q0 + b \cdot \q1\\[5mm]
   \onslide<3->{\q{$\Psi$}} &\onslide<3->{\only<-3|handout:0>{= a \cdot \q{00} + b \cdot \q{11}}\only<4-|handout:1>{= a \cdot \q{10} + b \cdot \q{01}}}
\end{align*}
      };
   }
   \onslide<5->{
      \node[above=8mm] (x) at(current page.south) {\usebeamerfont{subtitle}Verschränkung};
      \node[below=-2mm] at(x.south) {\color{gray}entanglement};
   }
\end{tikzpicture}
\end{frame}
% https://quantum-computing.ibm.com/composer/docs/iqx/guide/shors-algorithm

% https://quantum-computing.ibm.com/composer/docs/iqx/first-circuit

\def\Meter#1{\scope[black,line cap=round,line width=.35mm]
\fill (#1.south) circle[radius=.5mm];
\draw ([xshift=3mm]#1.south) arc(0:180:3mm);
\draw (#1.south) -- ++(55:4mm);
\endscope}
\begin{frame}{}
% we could use quantikz but for one circuit that should be potentially animated, i am just recreating a circuit based on the ibm style
% \newsavebox\tryItOutBox
% \setbox\tryItOutBox=\hbox{\fancyqr{https://quantum-computing.ibm.com/composer/files/new?initial=N4IgdghgtgpiBcIBCMA2qAEBlALhHcANCAI4QDOUCIA8gAoCiAcgIoCCWAshgEwB0ABgDcAHTABLMAGNUAVwAmMDCNJpxAIwCMfSVJWiwYkgCcYAcwwkA2jwC6BqaYtSrAFntiAFpasCP0gA8fP0IfTX9YCllTYNsMAFoAPgwXPwNI8mila3CE5JdwoRBiRXJHcQAHHHEAezBqEABfIA}}
\colorlet{@col}{gray!15}%
\Large\sbfamily
\setbox0=\hbox{\begin{tikzpicture}[@col,text=black,k/.style={fill,rectangle,minimum size=8mm,rounded corners=2pt}]
   \node (a) at (0,0)  {\small\llap{$q_0$}};
   \node (b) at (0,-1) {\small\llap{$q_1$}};
   \node (c) at (0,-2) {\small\llap{$c$}};
   \draw[lightgray] ([yshift=.25mm]a.east) coordinate (as) -- ++(.35\paperwidth,0);
   \draw[lightgray] ([yshift=.25mm]b.east) coordinate (bs) -- ++(.35\paperwidth,0);
   \draw[lightgray,double,double distance=.66mm] ([yshift=.25mm]c.east) coordinate (cs) -- ++(.35\paperwidth,0);
   \node[k] at([xshift=5mm]as) {H};
   \node[fill,circle,minimum size=8mm,outer sep=-2mm] (p) at([xshift=2cm]bs) {};
   \draw[black,line cap=round,line width=.65mm] (p.north) -- (p.south);
   \draw[black,line cap=round,line width=.65mm] (p.east) -- (p.west);
   \coordinate (pt) at([xshift=2cm]as);
   \draw[line width=1.25mm] ([yshift=2mm]p.north) -- (pt) coordinate (cnot-t);
   \fill (cnot-t) circle[radius=2mm];

   \node[k,outer sep=-2mm] (s1) at([xshift=32.5mm]as) {};
   \node[k,outer sep=-2mm] (s2) at([xshift=45mm]bs) {};
   \Meter{s1}
   \Meter{s2}
   \coordinate (c1) at([xshift=32.5mm]cs);
   \coordinate (c2) at([xshift=45mm]cs);
   \draw[-Kite,black] ([yshift=-2mm]s1.south) -- (c1);
   \draw[-Kite,black] ([yshift=-2mm]s2.south) -- (c2);
\end{tikzpicture}}%
\begin{tikzpicture}[@O]
   \onslide<2->{\node (b) at (current page.center) {\usebox0};}
   \onslide<3->{
      \node[above=8mm] (x) at(current page.south) {\usebeamerfont{subtitle}\href{https://quantum-computing.ibm.com/composer/docs/iqx/guide/shors-algorithm}{Shor-Algorithmus}};
      \path (b) -- (x) node[pos=.5,gray!15] (c) at(b.south) {\Large\faCaretDown};
   }
   \onslide<2->{
      \node[above right,T,scale=.7] at (current page.south west) {\mdseries\href{https://quantum-computing.ibm.com/composer/files/new?initial=N4IgdghgtgpiBcIBCMA2qAEBlALhHcANCAI4QDOUCIA8gAoCiAcgIoCCWAshgEwB0ABgDcAHTABLMAGNUAVwAmMDCNJpxAIwCMfSVJWiwYkgCcYAcwwkA2jwC6BqaYtSrAFntiAFpasCP0gA8fP0IfTX9YCllTYNsMAFoAPgwXPwNI8mila3CE5JdwoRBiRXJHcQAHHHEAezBqEABfIA}{\underline{\faLink\space Probiere es selbst!}}};
   }
\end{tikzpicture}
\end{frame}


\newsavebox\FinalPolarisationSlide
{%
\def\Sin#1{0.5*sin(deg(pi*#1))}
\def\Cos#1{0.5*cos(deg(pi*#1))}
\def\XShift{0}
\def\DrawUp#1#2{(\XShift,#1,#2)}
\def\DrawLeft#1#2{(\XShift+#1,0,#2)}
% \def\DrawT#1#2{(\XShift,#2,#1)}
\def\HelpLinesIf#1{
   \foreach \i in {-.5,-.4,...,5} {
      \pgfmathsetmacro\target{\Sin{\i}}
      \ifdim\target pt#10pt
         \draw[help] (\XShift,0,\i) -- \DrawT{\target}{\i};
      \fi
   }
}
\def\GridBack{
   \scope[canvas is xz plane at y=0]
      \draw[help,opacity=.65,step=.5] (-.5+\XShift-.001,-.75) grid (\XShift,5.25);
   \endscope
}
\def\GridFront{
   \scope[canvas is xz plane at y=0]
      \draw[help,opacity=.65,step=.5] (0.001+\XShift,-.75) grid (.5+\XShift,5.25);
   \endscope
}
\def\Pre#1{
   \def\XShift{#1}
   \HelpLinesIf<
   \GridBack
   \HelpLinesIf>
}
\newsavebox\Waves
\begin{frame}[c]
\hypertarget{@Polarisation}{}%
\begin{lrbox}\Waves
   \begin{tikzpicture}[z={(10:10mm)},x={(-65:6.5mm)},y=6.5mm,help/.style={gray!30,line cap=round},Arr/.style={{Kite[scale=.6]}-{Kite[scale=.6]}}]
      \onslide<2->{\draw[help] (-.5,0,6) -- ++(1,0,0);
      \draw[line cap=round,thick,Arr] (0,-.5,6)  -- ++(0,1,0);
      \let\DrawT\DrawUp
      \Pre0
      \draw[line cap=round,ultra thick, domain=-.5:5, samples=100, smooth] plot (0,{\Sin\x},\x);
      \GridFront}

      \onslide<3->{\draw[help] (3,-.5,6)  -- ++(0,1,0);
      \draw[line cap=round,thick,Arr] (2.5,0,6)  -- ++(1,0,0);
      \let\DrawT\DrawLeft
      \Pre3
      \GridFront
      \draw[line cap=round,ultra thick, domain=-.5:5, samples=100, smooth] plot ({\Sin\x+3},0,\x);}

      % \draw[line cap=round,thick,Arr] (-.5,0,6)  -- ++(1,0,0);
      \onslide<4->{\scope[canvas is xy plane at z=6]
         \draw[help] (-.5+6,0) -- ++(1,0);
         \draw[help] (6,-.5) -- ++(0,1);
         \draw[thick,{Kite[scale=.6]}-] (6.5,0) arc(0:360:.5);
      \endscope
      \let\DrawT\DrawUp
      \def\XShift{6}
      \GridBack
      \foreach \i in {-.5,-.4,...,5} {
         \draw[help] (\XShift,0,\i) -- ({\Sin\i+6},{\Cos\i},\i);
      }
      \draw[line cap=round,ultra thick, domain=-.5:5, samples=100, smooth] plot ({\Sin\x+6},{\Cos\x},\x);
      \GridFront}
   \end{tikzpicture}%
\end{lrbox}
\begin{tikzpicture}[@O]
   \node[above=8mm] (x) at(current page.south) {\usebeamerfont{subtitle}Polarisation};
   \node[scale=.9] at ([yshift=5mm]current page.center) {\usebox\Waves};
\end{tikzpicture}
\global\setbox\FinalPolarisationSlide=\hbox{\scalebox{.9}{\box\Waves}}
\end{frame}
}


% one time pads etc? es gibt suuuper viele Verfahren
% Binär <-> Polarisationen
% hier mit licht, da ist es recht anschaulich, geht aber mit diversen Qubits wie electron spin, ...
{
\newsavebox\CodeA \newsavebox\CodeB \newsavebox\CodeC \newsavebox\CodeD
\tikzset{Y/.style={very thick,Kite-Kite},N/.style={gray}}
\savebox\CodeA{\tikz{
   \draw[N] (1,-1) -- ++(0,2);
   \draw[Y] (0,0) -- ++(2,0);
}}
\savebox\CodeB{\tikz{
   \draw[N] (0,0) -- ++(2,0);
   \draw[Y] (1,-1) -- ++(0,2);
}}
\savebox\CodeC{\tikz{
   \draw[N] (0,0)++(45:1cm) -- ++(45:-2cm);
   \draw[Y] (0,0)++(-45:1cm) -- ++(-45:-2cm);
}}
\savebox\CodeD{\tikz{
   \draw[N] (0,0)++(-45:1cm) -- ++(-45:-2cm);
   \draw[Y] (0,0)++(45:1cm) -- ++(45:-2cm);
}}

\begin{frame}[c]{}
\begin{tikzpicture}[@O]
   \coordinate(@c) at (current page.center);
   \def\Shift{0cm}
   \only<4->{\def\Shift{(.25\paperwidth)}}
   \onslide<2->{
      \node (a) at ([xshift=-\Shift-.1\paperwidth]current page.center) {\usebox\CodeA};
      \node (b) at ([xshift=-\Shift+.1\paperwidth]current page.center) {\usebox\CodeB};
   }
   \onslide<3->{
      \node[below=2mm] at (a.south) {\usebeamerfont{subtitle}0};
      \node[below=2mm] at (b.south) {\usebeamerfont{subtitle}1};
   }

   \onslide<5->{
      \node (c) at ([xshift=\Shift-.1\paperwidth]current page.center) {\usebox\CodeC};
      \node (d) at ([xshift=\Shift+.1\paperwidth]current page.center) {\usebox\CodeD};
      \node[below=2mm] at (c.south) {\usebeamerfont{subtitle}0};
      \node[below=2mm] at (d.south) {\usebeamerfont{subtitle}1};
   }
\end{tikzpicture}
\end{frame}


% TODO: zwei Kanäle nötig: Quanten und herkömmlich
% => üblich für Quanten dass wir sie mit "normalem" kombinieren
\newsavebox\CodeSmall
\setbox\CodeSmall=\hbox{%
   \def\arraystretch{1.5}%
   \begin{tabular}{*4{>{\sbfamily\LARGE}m{\wd\CodeA}<{\centering\arraybackslash}}}
      \usebox\CodeA & \usebox\CodeB & \usebox\CodeC & \usebox\CodeD\\
      \strut0 & 1 & 0 & 1
   \end{tabular}%
}

\newsavebox\CodeSmallOne
\def\Opa#1{\tikz[opacity=.35]{\node[inner sep=0pt,outer sep=0pt] {#1};}}
\setbox\CodeSmallOne=\hbox{%
   \def\arraystretch{1.5}%
   \begin{tabular}{*4{>{\sbfamily\LARGE}m{\wd\CodeA}<{\centering\arraybackslash}}}
      \Opa{\usebox\CodeA} & \usebox\CodeB & \Opa{\usebox\CodeC} & \usebox\CodeD\\
      \strut\color{lightgray}0 & 1 & \color{lightgray}0 & 1
   \end{tabular}%
}

\newsavebox\SieveA \newsavebox\SieveB
\savebox\SieveA{\begin{tikzpicture}
   \fill[gray,rounded corners=4pt] (0,0) rectangle (1,1);
   \draw[line cap=round,line width=3pt,btdl@color@background] (.5,.2) -- ++(0,.6);
   \draw[line cap=round,line width=3pt,btdl@color@background] (.2,.5) -- ++(.6,0);
\end{tikzpicture}}

\savebox\SieveB{\begin{tikzpicture}
   \fill[gray,rounded corners=4pt] (0,0) rectangle (1,1);
   \draw[line cap=round,line width=3pt,btdl@color@background] (.5,.5)++(45:.3) -- ++(45:-.6);
   \draw[line cap=round,line width=3pt,btdl@color@background] (.5,.5)++(-45:.3) -- ++(-45:-.6);
\end{tikzpicture}}

\newsavebox\SuperSlide
% ohne eavesdrop: veröffentlihcen am ende ihre wahlen und vergleichen => nehmen nur die wo sie gleich gewählt haben.

\newsavebox\FinalSuperSlide

\begin{frame}{}
\hypertarget{@SuperSlide}{}%
\tikzset{@s/.style={}}%
\only<6|handout:0>{\tikzset{@s/.style={transparency group, opacity=.3}}}%
\begin{lrbox}\SuperSlide
\begin{tikzpicture}
   \coordinate (c) at (0,0);
   \onslide<2->{
      \node[scale=.75] (a) at ([xshift=-4.5cm]c) {\only<-5,7-|handout:1->{\usebox\pinguA}\only<6|handout:0>{\usebox\pinguAGray}};
      \node[below] (ad) at(a.south) {\only<6|handout:0>{\color{lightgray}}Alice};

      \node[scale=.75] (b) at ([xshift=4.5cm]c) {\only<2,6-|handout:1->{\usebox\pinguB}\only<3-5|handout:0>{\usebox\pinguBGray}};
      \node (bd) at(b.south|-ad.east) {\only<3-5|handout:0>{\color{lightgray}}Bob};
   }
   \scope[@s]
   \onslide<4->{
      \node[cloud,cloud ignores aspect, draw,above right=5mm,xshift=-5mm] (dice) at (a.north) {\usebox\Random};
      \node[above right,scale=.66] (a-sends) at (a.east) {\usebox\CodeB};
      \node[below,scale=.66] (ab-sends) at (a.east-|a-sends.south) {\usebox\CodeD};
   }
   \endscope
   \onslide<5->{
      \scope[shift={([xshift=1.66cm]a.east)}]
         \draw[thick,domain=0:3.5,samples=100,smooth] plot (\x,{0.5*sin(deg(2*pi*\x))});
      \endscope
   }
   \begin{onlyenv}<6->
      \node[above left] at (b.west) {\usebox\SieveA};
      \tikzset{@ss/.style={}}
      \node[below left] at (b.west) {\usebox\SieveB};
      \node[cloud,cloud ignores aspect, draw] (dice2) at ([xshift=-3.5mm]b.north|-dice.east) {\usebox\Random};
   \end{onlyenv}
   \begin{uncoverenv}<7->
      \draw[-Kite] ([yshift=-3mm]ad.south) to[edge node={node[above] {\(+~\times~+~+~\times~\)~\ldots}}] ([yshift=-3mm]bd.south|-ad.south);
      \draw[Kite-] ([yshift=-5mm]ad.south) to[edge node={node[below] {\(+~+~\times~+~\times~\)~\ldots}}] ([yshift=-5mm]bd.south|-ad.south);
   \end{uncoverenv}
\end{tikzpicture}
\end{lrbox}
\begin{tikzpicture}[@O]
   \node[yshift=-7.5mm] at (current page.center) {\usebox\SuperSlide};

   \node[below left=3.5mm,scale=.4,@s] at(current page.north east) {\only<-4,6-|handout:0>{\usebox\CodeSmall}\only<5|handout:1>{\usebox\CodeSmallOne}};
\end{tikzpicture}
% TODO: darüber sprechen, wi eman abhören erkennen kann (verändert polarisation, ... )
% Zudem viele andere möglichkeiten | verwerfen am ende mismathces
\global\setbox\FinalSuperSlide=\box\SuperSlide
\end{frame}
}
% https://www.qi.damtp.cam.ac.uk/files/PartIIQIC/QIC-6.pdf


{
\newsavebox\RecapSlideA \savebox\RecapSlideA{\PrintSlide\FinalCryptIntroSlide{@FinalCrypt}}
\newsavebox\RecapSlideB \savebox\RecapSlideB{\PrintSlide\FinalPolarisationSlide{@Polarisation}}
\newsavebox\RecapSlideC \savebox\RecapSlideC{\PrintSlide\FinalSuperSlide{@SuperSlide}}
\begin{frame}{}
\begin{uncoverenv}<2->
\begin{tikzpicture}[@O]
   \node[left=2mm] at(current page.center) {\usebox\RecapSlideA};
   \node[above right=2mm] at(current page.center) {\usebox\RecapSlideB};
   \node[below right=2mm] at(current page.center) {\usebox\RecapSlideC};
\end{tikzpicture}
\end{uncoverenv}
\end{frame}
}
