\newsavebox\Router \savebox\Router{\scalebox{.12}{\router{}}}
\tikzset{dot/.style={minimum width=8mm,minimum height=3mm,rounded corners=3pt,rectangle}}
% #4 label
\def\PlaceAt#1#2#3#4{%
   \node[outer sep=1pt,dot,#1] (#1) at([shift={(#2*\wd\Germany,#3*\ht\Germany+#3*\dp\Germany)}]germany.south west) {\only<-2|handout:1>{\smash{\raisebox{-4pt}{\clap{\usebox\Router}}}}};
   % #4
   \only<4-|handout:2>{\node[text=white] at (#1) {\tiny\sbfamily#4};}
}
\def\B#1{\only<5-|handout:2>{#1}}
\tikzset{
   X/.style n args=2{edge node={node [#2=-.5mm,sloped] {\tiny#1}}},
   A/.style={X={#1}{above}},
   B/.style={X={#1}{below}}
}

% TODO: frage von wo nach wo kommen wir wie einwerfen
{
% \On<3>{a}{fill=gray}
\newcommand<>\On[2]{\only#3{{%
   \globaldefs=1\relax%
   \tikzset{#1/.style={#2}}
}}}
\tikzset{
   current/.style={fill=black,draw=white,text=white},%
   unqueued/.style={fill=btdl@color@background,draw=gray,text=gray},
   queue/.style={fill=btdl@color@background,draw=black,text=black},
   startend/.style={unqueued,minimum size=5mm},
   visited/.style={fill=gray,draw=white,text=white}
}
\begin{frame}
\begin{tikzpicture}[@O,base/.style={},a/.style={base},b/.style={base},c/.style={base},d/.style={base},e/.style={base},f/.style={base},g/.style={base},h/.style={base},i/.style={base},j/.style={base}]
\begin{onlyenv}<-6|handout:1>
   \only<-5>{
      \node[inner sep=0pt,outer sep=0pt] (germany) at (current page.center) {\usebox\Germany};
      \node[above right,T] at(current page.south west) {\fontsize{3.5pt}{3.5pt}\selectfont
\href{https://commons.wikimedia.org/wiki/File:German_state_government_compositions.svg}{Deutschland basiert auf Grafik von \begin{CJK*}{UTF8}{gbsn}
沁水湾
\end{CJK*},}~~\href{https://creativecommons.org/licenses/by-sa/4.0/deed.en}{\ccbysa}};
   }
   \only<3-|handout:2>{\tikzset{dot/.style={minimum size=7mm,circle,fill=black,}}}
\onslide<2->{
   \PlaceAt{a}{.5}{.5}{e}
   \PlaceAt{b}{.8}{.75}{i}
   \PlaceAt{c}{.55}{.175}{b}
   \PlaceAt{d}{.25}{.15}{a}
   \PlaceAt{e}{.25}{.75}{h}
   \PlaceAt{f}{.7}{.42}{d}
   \PlaceAt{g}{.3}{.3}{c}
   \PlaceAt{h}{.625}{.7}{g}
   \PlaceAt{i}{.3}{.6}{f}
   \PlaceAt{j}{.45}{.8}{j}

   \graph[use existing nodes,edges={-}]{
      a ->[B=\B8] b;
      a ->[A=\B{7.5}] c;
      a ->[bend left=20,B={\B{10}}] d;
      a ->[A={\B{~~~~~~~~~~~7.5}}] e;
      a ->[A={\B3}] f;
      a ->[A={\B{5.5}}] g;
      h ->[B={\B6}] i ->[A=\B5] j ->[B=\B3] h;
      b ->[B={\B8}] f ->[B=\B6] c;
      % d -> e would be 15.5 but we choose a #fastlane
      d ->[bend left,A=\B{4}] e ->[A=\B{3}] j ->[A=\B6] b ->[A=\B2] h;
   };
}
\end{onlyenv}
\begin{onlyenv}<7-|handout:2-3>
   \tikzset{dot/.style={minimum size=7mm,circle,fill=black,text=white,font=\scriptsize\sbfamily}}
   \scope[shift={([xshift=-3cm,yshift=1.25cm]current page.center)},k/.style={font=\scriptsize,midway}]
   \only<8-|handout:3>{
      \tikzset{
         base/.style={unqueued},
         b/.style={startend},
         j/.style={startend},
      }
   }
   \foreach[count=\i from 11] \n in {e,d,,c,a,h,i,,,,,j,,g}{
      \ifx!\n!\else
         \On<\i-|handout:3>{\n}{queue}
      \fi
   }
   \On<10-12|handout:3>{b}{current}\On<13-|handout:3>{b}{visited}
   \On<13-17|handout:3>{e}{current}\On<18-|handout:3>{e}{visited}
   \On<18|handout:3>{d}{current}\On<19-|handout:3>{d}{visited}
   \On<19|handout:3>{c}{current}\On<20-|handout:3>{c}{visited}
   \On<20|handout:3>{a}{current}\On<21-|handout:3>{a}{visited}
   \On<21-22|handout:3>{h}{current}\On<23-|handout:3>{h}{visited}
   \On<23-24|handout:3>{i}{current}\On<25-|handout:3>{i}{visited}
   \On<25-|handout:3>{j}{current}

   % queue up there
   \scope[every node/.append style={outer xsep=4pt}]
   \only<9-|handout:3>{
      \node[dot,b,xshift=-3.65cm,below=1.5mm] (q1) at (current page.north) {b};
   }
   \only<11-|handout:3>{
      \node[dot,e,right] (q2) at (q1.east) {e};
   }
   \only<12-|handout:3>{
      \node[dot,d,right] (q3) at (q2.east) {d};
   }
   \only<14-|handout:3>{
      \node[dot,c,right] (q4) at (q3.east) {c};
   }
   \only<15-|handout:3>{
      \node[dot,a,right] (q5) at (q4.east) {a};
   }
   \only<16-|handout:3>{
      \node[dot,h,right] (q6) at (q5.east) {h};
   }
   \only<17-|handout:3>{
      \node[dot,i,right] (q7) at (q6.east) {i};
   }
   \only<22-|handout:3>{
      \node[dot,j,right] (q8) at (q7.east) {j};
   }
   \only<24-|handout:3>{
      \node[dot,g,right] (q9) at (q8.east) {g};
   }
   \only<9-|handout:3>{
      \draw[lightgray] ([yshift=-1.25cm]current page.north west) node[above right] {\tiny\sbfamily Warteschlange} -- ([yshift=-1.25cm]current page.north east);
   }
   \endscope

   \node[dot,a] (a) at (2,1) {a};
   \node[dot,b] (b) at (0,-2) {b};
   \node[dot,c] (c) at (0,0) {c};
   \node[dot,e] (e) at (2,-1) {e};
   \node[dot,d] (d) at (2,-3) {d};
   \node[dot,h] (h) at (4,1) {h};
   \node[dot,g] (g) at (6,-2) {g};
   \node[dot,i] (i) at (4,-3) {i};
   \node[dot,j] (j) at (4,-1) {j};
   \node[dot,f] (f) at (6,0) {f};
   \only<8-|handout:3>{
      \node[left,gray] at(b.west) {\faCaretRight};
      \node[left,gray] at(j.west) {\faCaretRight};
   }
   \scope[every path/.append style={line cap=round,gray}]
   \draw (a) -- (e) node[right,k] {10};
   \draw (a) -- (h) node[above,sloped,k] {4};
   \draw (b) -- (e) node[above,sloped,k] {7.5};
   \draw (b) -- (d) node[below,k] {6};

   \draw (d) -- (e) node[left,k] {3};
   \draw (d) -- (i) node[above,k] {8};

   \draw (e) -- (c) node[above,sloped,k] {5.5};
   \draw (e) -- (h) node[below,sloped,k] {7.5};
   \draw (e) -- (i) node[above,sloped,k] {8};

   \draw (f) -- (j) node[above,sloped,k] {5};
   \draw (f) -- (g) node[right,k] {6};

   \draw (g) -- (i) node[below,sloped,k] {2};
   \draw (g) -- (j) node[above,sloped,k] {3};

   \draw (h) -- (j) node[right,k] {3};

   \draw (i) -- (j) node[right,k] {6};
   \endscope
   \endscope
   \only<9-|handout:3>{
      \node[above=8mm] (@title) at(current page.south) {\usebeamerfont{subtitle}Breitensuche};
      \node[below=-1.75mm] at(@title.south) {\color{gray}{\sbfamily b}readth-{\sbfamily f}irst {\sbfamily s}earch};
   }
\end{onlyenv}
\end{tikzpicture}
\note[itemize]{
   \item Gaanz viele Router mit super viel Informationen (Städte,~\ldots)
}(2)
\end{frame}
% Optionen:
%    - Alle Routen | Aber: Schleifen
%    - alle Routen ohne schleifen | immer noch sehr viel
%    - abbrechen sobald wir wissen, dass es keine kürzere geben kann


\def\blk#1{\makebox[2em][c]{\strut#1}}
\def\Node#1#2#3{\strut#1\only<2->{%
   \nodepart{two}%
   % count number of formulas
   \gdef\weightcount{0}%
   \foreach[count=\i from 1] \x in {#2} {\xdef\weightcount{\i}}%
   \gdef\fromcount{0}%
   \foreach[count=\i from 1] \x in {#3} {\xdef\fromcount{\i}}%
   \blk{%
      \foreach[count=\i from 1] \range/\x in {#2}{%
         \ifnum\i=\weightcount \def\handoutNum{1}\else\def\handoutNum{0}\fi
         \only<\range|handout:\handoutNum>{\(\x\)\csxdef{#1Count}{\x}}%
      }%
   }%
   \quad
   \blk{%
      \foreach[count=\i from 1] \range/\x in {#3}{%
         \ifnum\i=\fromcount \def\handoutNum{1}\else\def\handoutNum{0}\fi%
         \only<\range|handout:\handoutNum>{\x}%
      }%
   }%
}}
\newsavebox\FinalDijkstra
\begin{frame}{}
\hypertarget{@Dijkstra}{}\def\n{---}%
\tikzset{base/.style={},a/.style={base},b/.style={base},c/.style={base},d/.style={base},e/.style={base},f/.style={base},g/.style={base},h/.style={base},i/.style={base},j/.style={base},dot/.style={minimum size=7mm,circle,fill=black,text=white,font=\scriptsize\sbfamily\boldmath,outer sep=0pt}}%
\begin{lrbox}\FinalDijkstra
\begin{tikzpicture}
\scope[k/.style={font=\scriptsize,midway},@scale/.style={}]
\only<2->{{
   \globaldefs=1\relax
   \tikzset{
      dot/.append style={shape=rectangle split,rounded corners=2pt,rectangle split parts=2pt,rectangle split part align={center}},
      base/.style={unqueued},
      b/.style={startend},
      j/.style={startend},
      @scale/.style={yscale=1.25,xscale=1.65},
   }
}}
% means more like "looked at for update
\foreach[count=\i from 6] \n in {e,e,e,d,d,d,,e,i,i,,c,a,h,i,,j,g,,,a,j,,,j,f}{
   \ifx!\n!\else
      \On<\i>{\n}{queue}
   \fi
}
\On<4-11>{b}{current}\On<12->{b}{visited}
\On<12-15>{d}{current}\On<16->{d}{visited}
\On<16-20>{e}{current}\On<21->{e}{visited}
\On<21>{c}{current}\On<22->{c}{visited}
\On<22-24>{i}{current}\On<25->{i}{visited}
\On<25-27>{h}{current}\On<28->{h}{visited}
\On<28-30>{g}{current}\On<31->{g}{visited}
\On<31>{a}{current}\On<32->{a}{visited}
\On<32->{j}{current}% \On<33->{g}{visited}
\scope[@scale]
% "Lauflängenkompression" could have been helpful xD
\node[dot,a] (a) at (2,1)  {\Node{a}{3-17/\infty,18-/17.5}{3-17/\n,18-/e}};
\node[dot,b] (b) at (0,-2) {\Node{b}{3-4/\infty,5-/0}{3-/\n}};
\node[dot,c] (c) at (0,0)  {\Node{c}{3-16/\infty,17-/13}{3-16/\n,17-/e}};
\node[dot,e] (e) at (2,-1) {\Node{e}{3-6/\infty,7-/7.5}{3-7/\n,8-/b}};
\node[dot,d] (d) at (2,-3) {\Node{d}{3-9/\infty,10-/6}{3-10/\n,11-/b}};
\node[dot,h] (h) at (4,1)  {\Node{h}{3-18/\infty,19-/15}{3-18/\n,19-/e}};
\node[dot,g] (g) at (6,-2) {\Node{g}{3-23/\infty,24-/16}{3-23/\n,24-/i}};
\node[dot,i] (i) at (4,-3) {\Node{i}{3-14/\infty,15-/14}{3-14/\n,15-/d}};
\node[dot,j] (j) at (4,-1) {\Node{j}{3-22/\infty,23-26/20,27-/18}{3-22/\n,23-26/i,27-/h}};
\node[dot,f] (f) at (6,0)  {\Node{f}{3-29/\infty,30-/22}{3-29/\n,30-/g}};
\only<2->{\foreach \n in {a, ..., j} {
   % get the additional split | text split did not work
   \draw[\n] ([yshift=0mm]\n.center) -- (\n.south);
}}
\endscope
\pgfonlayer{background}
\scope[every path/.append style={line cap=round,gray}]
\draw (a) -- (e) node[right,k] {10};
\draw (a) -- (h) node[above,sloped,k] {4};
\draw (b) -- (e) node[above,sloped,k] {7.5};
\draw (b) -- (d) node[below,k] {6};

\draw (d) -- (e) node[left,k] {3};
\draw (d) -- (i) node[above,k] {8};

\draw (e) -- (c) node[above,sloped,k] {5.5};
\draw (e) -- (h) node[below,sloped,k] {7.5};
\draw (e) -- (i) node[above,sloped,k] {8};

\draw (f) -- (j) node[above,sloped,k] {5};
\draw (f) -- (g) node[right,k] {6};

\draw (g) -- (i) node[below,sloped,k] {2};
\draw (g) -- (j) node[above,sloped,k] {3};

\draw (h) -- (j) node[right,k] {3};

\draw (i) -- (j) node[right,k] {6};
\only<33->{ \draw[-Circle,thick,black] (j) -- (h); }
\only<34->{ \draw[-Circle,thick,black] (h) -- (e); }
\only<35->{ \draw[-Circle,thick,black] (e) -- (b); }
\endscope
\endpgfonlayer
\endscope
\only<1->{
   \node[left,gray] at(b.west) {\faCaretRight};
   \node[left,gray] at(j.west) {\faCaretRight};
}
\end{tikzpicture}
\end{lrbox}
\begin{tikzpicture}[@O]
   \node[scale=.9] at ([yshift=-2mm]current page.center) {\usebox\FinalDijkstra};
\only<5->{
   % TODO: access \csxdef{#1Count}{\x}
   \draw[lightgray] ([yshift=-1.25cm]current page.north west) node[above right] {\tiny\sbfamily Prioritätswarteschlange} -- ([yshift=-1.25cm]current page.north east);
   \foreach[count=\i from 1] \l in {a, ..., j} {
      \node[dot,rectangle split,draw, rectangle split parts=2,rectangle split horizontal,below=4mm,rounded corners=2pt,scale=.85,rectangle split part align=center,\l,minimum size=3mm]
         at ([xshift=1.25cm*(\i-9/2)+2mm]current page.north)
      {%
         \makebox[1em][c]{\strut\raisebox{-1pt}{\l}\strut}%
         \nodepart{two}%
         \makebox[1.75em][c]{\strut\raisebox{-1pt}{\(\csuse{\l Count}\)}\strut}%
      };
   }
}
\only<36->{
   \node[above=3mm] at(current page.south) {\usebeamerfont{subtitle}Dijkstra-Algorithmus};
}
\end{tikzpicture}
\global\setbox\FinalDijkstra=\box\FinalDijkstra
\end{frame}

\def\PlaceAt#1#2#3#4{%
   \node[outer sep=1pt,dot,#1] (#1) at([shift={(#2*\wd\Germany,#3*\ht\Germany+#3*\dp\Germany)}]germany.south west) {};
   \node[text=white] at (#1) {\tiny\sbfamily#4};
}
\def\B#1{#1}%
\begin{frame}
\tikzset{base/.style={},a/.style={base},b/.style={base},c/.style={base},d/.style={base},e/.style={base},f/.style={base},g/.style={base},h/.style={base},i/.style={base},j/.style={base},dot/.style={minimum size=7mm,circle,fill=black,text=white,font=\scriptsize\sbfamily\boldmath,outer sep=0pt}}%
\begin{tikzpicture}[@O]
      \node[inner sep=0pt,outer sep=0pt] (germany) at (current page.center) {\usebox\Germany};
      \node[above right,T] at(current page.south west) {\fontsize{3.5pt}{3.5pt}\selectfont
   \href{https://commons.wikimedia.org/wiki/File:German_state_government_compositions.svg}{Deutschland basiert auf Grafik von \begin{CJK*}{UTF8}{gbsn}
   沁水湾
   \end{CJK*},}~~\href{https://creativecommons.org/licenses/by-sa/4.0/deed.en}{\ccbysa}};
   \tikzset{dot/.style={minimum size=5mm,circle,fill=black,}}
   \PlaceAt{a}{.5}{.5}{e}
   \PlaceAt{b}{.8}{.75}{i}
   \PlaceAt{c}{.55}{.175}{b}
   \PlaceAt{d}{.25}{.15}{a}
   \PlaceAt{e}{.25}{.75}{h}
   \PlaceAt{f}{.7}{.42}{d}
   \PlaceAt{g}{.3}{.3}{c}
   \PlaceAt{h}{.625}{.7}{g}
   \PlaceAt{i}{.3}{.6}{f}
   \PlaceAt{j}{.45}{.8}{j}

   \graph[use existing nodes,edges={-}]{
      a ->[B=\B8] b;
      a ->[A=\B{7.5},Kite-] c;
      a ->[bend left=20,B={\B{10}}] d;
      a ->[A={\B{~~~~~~~~~~~7.5}},-Kite] e;
      a ->[A={\B3}] f;
      a ->[A={\B{5.5}}] g;
      h ->[B={\B6}] i ->[A=\B5] j ->[B=\B3] h;
      b ->[B={\B8}] f ->[B=\B6] c;
      % d -> e would be 15.5 but we choose a #fastlane
      d ->[bend left,A=\B{4}] e ->[A=\B{3},-Kite] j ->[A=\B6] b ->[A=\B2] h;
   };
\end{tikzpicture}
\end{frame}

% algorithms did blow up the pdflatex main memory size
{\def\mark#1{{\sbfamily#1}}

\newsavebox\FinalPseudo
\begin{frame}{}
\hypertarget{@Pseudocode}{}%
\begin{lrbox}{\FinalPseudo}%
\begin{minipage}{.7\linewidth}
\scriptsize\mark{Eingabe:} Graph \(G = (V, E)\), Start- und Endknoten \(s, e \subseteq V\)\bigskip\\
Erzeuge Prioritätswarteschlange \(Q\)\smallskip\\
\mark{Für alle} \(v \in V\) \mark{tue:}\\
\null\qquad \(v.\text{Kosten} \gets \infty\)\\
\null\qquad \(v.\text{Vorgänger} \gets \text{?}\)\\
\null\qquad Füge \(v\) zu \(Q\) hinzu\smallskip\\
\(s.\text{Kosten} \gets 0\)\smallskip\\

\mark{Solange} \(Q\) nicht leer ist, \mark{tue:}\\
\null\qquad \(u \gets \text{Wähle}~v\) mit minimalen Kosten aus \(Q\)\smallskip\\
\null\qquad \mark{Wenn} \(u\) das gesuchte \(e\) ist, \mark{dann:}\\
\null\qquad\qquad \mark{Gib} Pfad von \(e\) über Vorgänger \mark{zurück}\smallskip\\
\null\qquad Entferne \(u\) aus \(Q\)\smallskip\\
\null\qquad \mark{Für alle} Nachbarn \(v\) von \(u\) \mark{tue:}\\
\null\qquad\qquad \(a \gets u.\text{Kosten} + \text{Kosten der Kante } \{u, v\}\)\\
\null\qquad\qquad \mark{Wenn} \(a < v.\text{Kosten}\), \mark{dann:}\\
\null\qquad\qquad\qquad \(v.\text{Kosten} \gets a\)\\
\null\qquad\qquad\qquad \(v.\text{Vorgänger} \gets u\)\smallskip\\
\mark{Gib zurück}, dass kein Pfad gefunden wurde
\end{minipage}
\end{lrbox}%
\begin{tikzpicture}[@O]
   \node[align=left] at(current page.center) {%
      \usebox\FinalPseudo
   };
   \node[above right,gray,scale=.5] at(current page.south west) {\tiny Dem hier gezeigten Pseudocode fehlen Optimierungen, die in der vorherigen Animation verwendet wurden.};
   % z.b., dass wir hier versuchen immer jeden Nachbar zu aktualisieren
\end{tikzpicture}
\global\setbox\FinalPseudo=\box\FinalPseudo
\end{frame}

\begin{frame}[c]{}
\begin{tikzpicture}[@O]
   \node[align=left,text width=.7\linewidth] at(current page.center) {%
      \begin{itemize}
         \itemsep16pt
         \item<2-> Korrektheitsbeweise
         \item<3-> Laufzeitanalysen
         \item<4-> Optimierungen und alternative Ansätze
         \item<5-> Implementierung
      \end{itemize}
   };
\end{tikzpicture}
\end{frame}

}
% O(Kantenanzahl + Knoteanzahl^2) = O(Knotenanzahl^2)
}


{
\newsavebox\InfRecapSlideA \savebox\InfRecapSlideA{\PrintSlide\FinalInfIntro{@FinalWorldIntro}}
\newsavebox\InfRecapSlideB \savebox\InfRecapSlideB{\PrintSlide\FinalDijkstra{@Dijkstra}}
\newsavebox\InfRecapSlideC \savebox\InfRecapSlideC{\PrintSlide\FinalPseudo{@Pseudocode}}
\begin{frame}{}
\begin{uncoverenv}<2->
\begin{tikzpicture}[@O]
   \node[left=2mm] at(current page.center) {\usebox\InfRecapSlideA};
   \node[above right=2mm] at(current page.center) {\usebox\InfRecapSlideB};
   \node[below right=2mm] at(current page.center) {\usebox\InfRecapSlideC};
\end{tikzpicture}
\end{uncoverenv}
\end{frame}
}