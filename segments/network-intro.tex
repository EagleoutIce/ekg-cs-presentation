\input{data/world.tex}
\newsavebox\World
\newsavebox\Home
\savebox\Home{\colorlet{black}{darkgray}\begin{tikzpicture}[shift={(-2cm,2cm)}, router size/.style={scale=.2}, tiny router size/.style={scale=.1},link/.style={},gray]
   \draw[rounded corners=2pt] (-3.55,-.25) -| ++ (-2,1.5) -- ++(-.25,0) -- ++(1.25,.5) -- ++(1.25,-.5) -| (-3.55,-.25) -- cycle;

   % ignore for old reasons :P
   \node[tiny router size] (homenet1) at (-4,0) {\router{}};
   \node[server,scale=.75] (homepc1) at ($(homenet1)+(0,.75)$) {};
   % \node[server,scale=.75] (homepc2) at ($(homepc1)+(-.85,.125)$) {};

   \node[rack switch,scale=.5] (homeroute1) at ($(homenet1)+(-.85,0)$) {};
   \node[above=2mm,xshift=-1.5mm,scale=.3,xscale=-1] at (homeroute1.north) {\usebox\pinguA};
   % "wifi":
   % \node at([yshift=.3cm,xshift=-.075cm]homeroute1) {\faWifi};
   % \node[below=.25cm] at ([xshift=-.5cm]homenet1) {home network};
   % connect the home to the web :P
   \draw[link] (homepc1) -- (homenet1) -- (homeroute1);
\end{tikzpicture}}
\begin{frame}
\only<1|handout:0>{\savebox\World{\tikz{\WORLD}}}
\only<2->{%
   \savebox\World{%
      \tikz{%
         \WORLD[hl/.style={fill=lightgray},WORLD map/state/.style={fill=lightgray!20},WORLD map/Germany/.style={hl},WORLD map/UnitedStatesOfAmerica/.style={hl},WORLD map/China/.style={hl}]%
         \onslide<3->{
            \colorlet{@col}{black}
            \only<4->{\colorlet{@col}{gray}}
            \draw[-Kite,@col] ([xshift=-.25pt,yshift=-3.25pt]Germany.center) to[bend right=25] ([xshift=1cm,yshift=-4mm]UnitedStatesOfAmerica.center);
            \draw[-Kite,@col] ([xshift=-.25pt,yshift=-3.25pt]Germany.center) to[bend left=25] (China.center);
         }
         \only<6|handout:0>{
            \draw[thick] ([xshift=-.25pt,yshift=-3.25pt]Germany.center) -- ++(-1mm,1mm) coordinate (@t); % -- (Netherlands.center);
         }
         % traceroute -e -I www.google.com
         % traceroute to www.google.com (142.250.181.228), 30 hops max, 60 byte packets
         %  1  _gateway (192.168.0.1)  0.938 ms  1.009 ms  1.191 ms
         %  2  ip-081-210-178-001.um21.pools.vodafone-ip.de (81.210.178.1)  15.015 ms  24.059 ms  25.599 ms
         %  3  ip-081-210-178-000.um21.pools.vodafone-ip.de (81.210.178.0)  25.673 ms de-fra01b-rc2-ae-36-0.aorta.net (84.116.197.181) <MPLS:L=960499,E=0,S=1,T=1>  33.986 ms  33.977 ms
         %  4  de-bfe18a-rt01-lag-1.aorta.net (84.116.190.34)  29.920 ms  31.491 ms  33.834 ms
         %  5  74.125.48.122 (74.125.48.122)  39.709 ms  41.292 ms  41.357 ms
         %  6  142.251.65.75 (142.251.65.75)  41.344 ms * *
         %  7  142.250.237.177 (142.250.237.177)  34.864 ms * *
         %  8  * * *
         %  9  * * *
         % 10  * * *
         % 11  * * fra16s56-in-f4.1e100.net (142.250.181.228)  20.571 ms

         \only<7|handout:0> {
            % address:        Liberty Global B.V.
            % address:        Boeing Avenue 53
            % address:        1119PE Schiphol-Rijk
            % address:        Netherlands
            % https://www.whois.com/whois/84.116.190.34
            \draw[thick] ([xshift=-.25pt,yshift=-3.25pt]Germany.center) -- ++(-1mm,1mm) -- (Netherlands.center)  coordinate (@t);
         }
         \only<8|handout:0> {
            % Newbury
            % address:        Vodafone House, The Connection
            % address:        RG14 2FN
            % address:        Newbury
            % address:        UNITED KINGDOM
            % https://www.whois.com/whois/195.2.26.93
            \draw[thick] ([xshift=-.25pt,yshift=-3.25pt]Germany.center) -- ++(-1mm,1mm) -- (Netherlands.center) -- (UnitedKingdom.center) coordinate (@t);
         }
         \only<9-> {
            % Address:        1600 Amphitheatre Parkway
            % City:           Mountain View
            % StateProv:      CA
            % PostalCode:     94043
            % Country:        US
            % https://www.whois.com/whois/142.251.65.75
            \draw[thick] ([xshift=-.25pt,yshift=-3.25pt]Germany.center) -- ++(-1mm,1mm) -- (Netherlands.center) -- (UnitedKingdom.center) -- ([yshift=-5mm,xshift=-.25mm]UnitedStatesOfAmerica.center) coordinate (@t);
            \node[left=3.5mm] at(@t) {\llap{\url{www.google.com}}};
         }
         % traceroute -e -I gaj.sh.gov.cn
         % traceroute to gaj.sh.gov.cn (175.6.201.209), 30 hops max, 60 byte packets
         %  1  _gateway (192.168.0.1)  1.388 ms  1.466 ms  2.973 ms
         %  2  ip-081-210-178-001.um21.pools.vodafone-ip.de (81.210.178.1)  13.688 ms  21.656 ms  23.198 ms
         %  3  ip-081-210-178-000.um21.pools.vodafone-ip.de (81.210.178.0)  23.186 ms de-fra01b-rc2-ae-36-0.aorta.net (84.116.197.181) <MPLS:L=960499,E=0,S=1,T=1>  31.326 ms  31.224 ms
         %  4  de-bfe18a-rt01-lag-1.aorta.net (84.116.190.34)  31.149 ms  31.140 ms  31.194 ms
         %  5  ae8-100-tcr1.fnt.cw.net (195.2.26.93)  27.553 ms  29.138 ms  30.963 ms
         %  6  ae34-pcr1.fnt.cw.net (195.2.31.38) <MPLS:L=90982,E=0,S=1,T=1>  111.836 ms *  111.540 ms
         %  7  ae36-xcr1.ltw.cw.net (195.2.2.73) <MPLS:L=66670,E=0,S=1,T=1>  39.553 ms * *
         %  8  * * *
         %  9  * * *
         % 10  * * *
         % 11  * * *
         % 12  * * *
         % 13  * 202.97.65.90 (202.97.65.90)  312.495 ms *
         % 14  202.97.24.94 (202.97.24.94)  327.133 ms  328.850 ms  328.962 ms
         % 15  61.137.11.50 (61.137.11.50)  321.145 ms  322.674 ms *
         % 16  175.6.255.82 (175.6.255.82)  330.275 ms  326.986 ms  328.462 ms
         % 17  * * *
         % 18  * * *
         % 19  * * *
         % 20  * * *
         % 21  175.6.201.209 (175.6.201.209)  329.045 ms  332.555 ms  338.253 ms
         \only<10->{
            % https://www.whois.com/whois/84.116.190.34 Netherlands
            % https://www.whois.com/whois/195.2.26.93 UK (multiple)
            % https://www.whois.com/whois/202.97.65.90 China Telecom Beijing/Peking
            % https://www.whois.com/whois/61.137.11.50 Chinanet hunan Hunan
            % ...
            % https://www.whois.com/whois/175.6.201.209 Chinanet Bejing
            \draw[thick] ([xshift=-.25pt,yshift=-3.25pt]Germany.center) -- ++(-1mm,1mm) -- (Netherlands.center) -- (UnitedKingdom.center) -- ([xshift=8.5mm,yshift=3mm]China.center) coordinate (@c) -- ([xshift=4.5mm,yshift=-4mm]China.center);
            \fill[circle] (@c) circle[radius=.75mm];
            \node[right=3.5mm] at(@c) {\url{gaj.sh.gov.cn}};
         }
         \only<6->{%
            \fill[circle] (@t) circle[radius=.75mm];
         }
      }%
   }%
}%
\begin{tikzpicture}[@O]
   \node[below left,gray,opacity=.33] at (current page.north east) {\scriptsize\sbfamily Vereinfacht};
   \node at (current page.center) {\resizebox{.8\paperwidth}!{\usebox\World}};
   \node[above left,T] at (current page.south east) {\fontsize{3.5pt}{3.5pt}\selectfont\href{https://tex.stackexchange.com/a/430463}{Weltkarte basiert auf Code von Hendrik Seliger}~~\href{https://creativecommons.org/licenses/by-sa/4.0/deed.en}{\ccbysa}};

   \only<4->{
      \node[above right=3mm] (germany) at (current page.south west) {\scalebox{.6}{\usebox\Germany}};
   \onslide<5->{\fill ([xshift=2.15cm*.6,yshift=1.3cm*.6]germany.south west) circle[radius=.75mm] coordinate (home-on-map);}
   \node[above right,T] at(current page.south west) {\fontsize{3.5pt}{3.5pt}\selectfont
\href{https://commons.wikimedia.org/wiki/File:German_state_government_compositions.svg}{Deutschland basiert auf Grafik von \begin{CJK*}{UTF8}{gbsn}
沁水湾
\end{CJK*},}~~\href{https://creativecommons.org/licenses/by-sa/4.0/deed.en}{\ccbysa}};
   }
   \only<5->{
      \node[above=3mm,xscale=-1] (home) at (current page.south) {\usebox\Home};
      % TODO: skipping verteiler
      \draw[densely dashed] (home-on-map) -- ([xshift=3mm,yshift=4.25mm]home.south east);
   }
   \only<6->{
      \fill ([xshift=1.8cm*.6,yshift=2.81cm*.6]germany.south west) circle[radius=.75mm] coordinate (frankfurt-on-map) node[right=1mm,align=left,scale=.5] {\textbf{Frankfurt}\\[-1mm]am Main};
      \draw[rounded corners=1pt,densely dotted] (home-on-map) -- ++(5mm,3mm) coordinate (c1) -- ++(-6mm,2mm) coordinate (c2) -- ++(-2mm,3mm)  coordinate (c3) -- (frankfurt-on-map);
      \foreach \i in {1,2,3} {
         \fill [gray] (c\i) circle[radius=.5mm];
      }
   }
   \only<7->{
      \draw[-Kite,densely dashed] (frankfurt-on-map) to[out=110,in=-70] ++(-.65cm,1cm);
   }

\end{tikzpicture}
\end{frame}
% und tatsächlich müssten wir eig erstmal wissen wo die server tatsächlich liegen.


% nur kurz als einblick, machen wir aber alles nicht :P


{
% #4 will be ighlight for aplayer
\colorlet{aplfill}{btdl@color@background}
\def\DrawLayers#1#2#3#4{
      \foreach[count=\y] \layer in {#2} {
         % \ifnum\y=#4 \relax \colorlet{aplfill}{green!15} \fi
         \draw[fill=aplfill,draw=black] ($(#1)+(0,0.5*\y)$) rectangle ++(#4,0.5) node[midway,centered,rectangle,minimum width=2cm,minimum height=0.5cm,inner sep=0pt,outer sep=0pt] (#3\layer) {\strut\layer};
      }
}

\begin{frame}
\colorlet{@hla}{gray!15}%
\colorlet{@hlb}{gray!15}%
\only<3->{\colorlet{@hla}{gray!30}\colorlet{@hlb}{gray!5}}
\begin{uncoverenv}<2->
% horrible tikz figure i used in a lecture recap, sloppily adapted to fit on this slide
\begin{tikzpicture}[every node/.style={font={\footnotesize\sffamily}, scale=.65}, router size/.style={scale=0.2}, tiny router size/.style={scale=0.1},scale=0.65,link/.style={thick},every path/.append style={line cap=round, line join=round}, @O] % very thick,green!75

   % split tags
   \draw[densely dashed] ([xshift=-.2\paperwidth]current page.north) -- ([xshift=-.3\paperwidth]current page.south);
   \draw[densely dashed] ([xshift=.25\paperwidth]current page.north) -- ([xshift=.15\paperwidth]current page.south);

   % labels
   \node[below=2mm,font=\large\sffamily] at (current page.north) {Network Core};
   \path (current page.north west) -- ([xshift=-.2\paperwidth]current page.north) node[pos=.5,below=2mm,font=\large\sffamily] {Network Edge};
   \path (current page.north east) -- ([xshift=.25\paperwidth]current page.north) node[pos=.5,below=2mm,font=\large\sffamily] {Network Edge};


   % todo: fix nested tikzpic via router, i'm lazy right now [poor flo from 2019]
   %% Regional ISP
   \scope[shift={([xshift=-2cm,yshift=-2cm]current page.center)}]
   \fill[@hla] (0,-0.85) circle (1.85);
   \node[router size] (regisp1) at (0,0) {\router{}};
   \node[router size] (regisp2) at ($(regisp1) + (-1,-1)$) {\router{}};
   \node[router size] (regisp3) at ($(regisp1) + (1,-1)$) {\router{}};
   \node[below] at($(regisp1) + (0,-1.5)$) {reg. ISP};
   \draw[link] (regisp1) -- (regisp2) -- (regisp3) -- (regisp1); % connect regional isp

   %% global ISP, seems to be somewhat 'crazy' :P
   \coordinate (globcenter) at (3,3.5);
   \fill[@hla] (globcenter) circle (2.125);
   \node[router size] (globisp1) at (2,3) {\router{}};
   \node[router size] (globisp2) at ($(globisp1) + (-0.25,0.8)$) {\router{}};
   \node[router size] (globisp3) at ($(globisp1) + (1.75,1.15)$) {\router{}};
   \node[router size] (globisp4) at ($(globisp1) + (2.25,0.15)$) {\router{}};
   \node[above] at ([yshift=1cm]globcenter) {glob. ISP~~};
   \draw[link] (globisp1) edge (globisp2) (globisp2) edge (globisp3) (globisp3) edge (globisp4) (globisp4) edge  (globisp1) (globisp1) edge (globisp3); % connect regional isp
   % dotted 'world' connection:
   \draw[densely dashed] (globisp4) -- ++(1.75,0);
   % connect global and local ISP
   \draw[link] (regisp1) -- (globisp1) (regisp3) -- (globisp4);

   \foreach \i in {3}{% ,3,4,1
         \relax \DrawLayers{globisp\i.north}{physical,link,network}{isp\i-}{2}
         \draw[fill=btdl@color@background,draw=black,line join=bevel] ([yshift=2mm]globisp\i.north) -- (isp\i-physical.south west) -- (isp\i-physical.south east) -- cycle;
   }

   %% home network
   % let's make a house
   \fill[@hlb] (-5.5,-1.25) -| ++ (-2,1.5) -- ++(-0.25,0) -- ++(1.25,0.5) -- ++(1.25,-0.5) -| cycle;

   \node[tiny router size] (homenet1) at (-6,-1) {\router{}};
   \node[server,scale=0.75] (homepc1) at ($(homenet1)+(0,0.75)$) {};
   \node[server,scale=0.75] (homepc2) at ($(homepc1)+(-0.85,0.125)$) {};

   \node[rack switch,scale=0.5] (homeroute1) at ($(homenet1)+(-0.85,0)$) {};
   % "wifi":
   \node at([yshift=0.3cm,xshift=-0.075cm]homeroute1) {\faWifi};
   \node[below=0.25cm] at ([xshift=-0.5cm]homenet1) {Heimnetzwerk};
   % connect the home to the web :P
   \draw[link] (homepc1) -- (homenet1) (homenet1) -- (homeroute1);


   \coordinate (dslamcenter) at ([xshift=1.25cm,yshift=-0.25cm]homenet1);


   \begin{scope}[scale=0.4]
         \draw[densely dashed,fill=btdl@color@background] ([xshift=2.25cm,yshift=-0.25cm]dslamcenter) -| ++ (-3,2.5) -- ++(-0.25,0) -- ++(1.75,1) -- ++(1.75,-1) -| cycle;

         \filldraw[left color=btdl@color@background,right color=gray,line join=round] (dslamcenter) -- ++(0.4,0) -- ++(0.5,0.25) -- ++(0.2,0) -- ++(0,1) -- ++(-0.2,0) -- ++(-0.5,0.25) -- ++(-0.4,0)--cycle;
         \draw[fill=gray,line join=round] (dslamcenter)++(0,1.5) -- ++(0.25,0.5) -- ++(0.4,0) -- ++(0.5,-0.25) -- ++(0.2,0) -- ++(-0.25,-0.5) -- ++(-0.2,0) -- ++(-0.5,0.25) -- ++(-0.4,0);
         \draw[fill=gray,line join=round] (dslamcenter)++(1.1,0.25) -- ++(0.25,0.5) -- ++(0,1) -- ++(-0.25,-0.5) -- cycle;
         \draw[-{Kite[scale=.4]}] (dslamcenter)++(0.1,0.25)-- ++(0.2,0) -- ++(0.5,0.25) -- ++(0.2,0);
         \draw[-{Kite[scale=.4]}] (dslamcenter)++(0.1,1.25)-- ++(0.2,0) -- ++(0.5,-0.25) -- ++(0.2,0);
         \draw[-{Kite[scale=.4]}] (dslamcenter)++(0.1,0.75)-- ++(0.9,0);
         % 1.25,3.25
         \node[above] at([xshift=0.75cm,yshift=3.25cm]dslamcenter) {DSLAM};

         \coordinate (dslamexit) at ([xshift=1.35cm,yshift=0.75cm]dslamcenter);

         % \node[above=2cm] at ([xshift=0.75cm]dslamcenter) {DSLAM};
         \draw[link] (dslamexit) -- ([xshift=-4.2cm]dslamexit-|regisp1);

         \draw (dslamexit)++(0,0.75) -- ++(1.25,0) node[right] {Tel. Net.};
   \end{scope}
   \draw[] (homenet1) -- ++(1.25cm,0);


   %% institution network, we will do it tiny for 'consistency reasons' :P

   \coordinate (instinit) at ($(regisp1)+(9.75,-2.5)$);
   \fill[@hlb] (instinit)++(2,-1.25) rectangle ++(-6,2);
   \node[tiny router size] (instrout1) at (instinit) {\router{}};
   \node[tiny router size] (instrout2) at ($(instrout1) + (-0.95,-0.5)$) {\router{}};
   \node[tiny router size] (instrout3) at ($(instrout1) + (0.5,-0.5)$) {\router{}};
   \draw[link] (instrout1) -- (instrout2) -- (instrout3) -- (instrout1); % connect regional isp

   \node[server,scale=0.75] (instserv1) at ($(instrout3)+(1.25,0.5)$) {};
   \node[server,scale=0.75] (instserv2) at ($(instserv1)+(-0.285,0)$) {};
   % you will recive
   \node[server,scale=0.75] (instserv3) at ($(instserv2)+(-0.285,0)$) {};

   \draw[link] (instrout3) -| (instserv1) (instrout3) -| (instserv2) (instrout3) -| (instserv3);

   \node[below=0.25cm] at ([xshift=-0.25cm]instrout3) {Institutsnetz};


   % the cheap wifi i provide them (somehow i, well... done something wrong with the naming. only needed twice, nvmd):
   \node[rack switch,scale=0.5] (instroute1) at ($(instrout1)+(-0.85,0)$) {};

   \node[server,scale=0.75] (instpc1) at ($(instroute1)+(-0.75,0)$) {};

   \foreach \i [remember=\i as \lasti (initially 1)] in {2,...,4} {
         \node[server,scale=0.75] (instpc\i) at ($(instpc\lasti)+(-0.6,0)$) {};
   }


   \node[server,scale=0.5] (instpc5) at ($(instpc1)+(-0.125,-0.75)$) {};
   \node[server,scale=0.5] (instpc6) at ($(instpc3)+(-0.125,-0.75)$) {};

   \draw[link] (instpc5) -- (instrout2);

   % "wifi":
   \node at([yshift=0.3cm,xshift=-0.075cm]instroute1) {\faWifi};

   % connect the instituion to the regional isp
   \draw[link] (instroute1) -- (instrout1) |- (regisp3);

   %% mobile world:
   \coordinate (mobinit) at (-4,3);
   \fill[@hlb] (mobinit)+(-1,1) circle (2.125);
   \draw (mobinit)+(-1,2.5) node {Mobilnetzwerk};
   \node[tiny router size] (mobilenet1) at (mobinit) {\router{}};

   % well, i have no shortcut for a wifi-station or something similar, so let's create it on the spot:
   \begin{scope}[every path/.style={very thick},scale=0.25]
         \coordinate (BaseStart) at ($(mobilenet1)+(-3.25,1.5)$);
         %trapezoid base
         \draw[line join=round] (BaseStart) -- ++(1.5,0.75) -- ++(1.5,-0.75) -- ++(-1.5,-0.75) -- cycle;%
         %draw tower:
         \draw (BaseStart) -- ++(1.5,3.5) -- ++(1.5,-3.5) (BaseStart)++(1.5,-0.75) -- ++(0,4.25);
         %draw lines:

         \coordinate (BottomPoint) at ($(BaseStart)+(1.5,-0.75)$);
         \coordinate (BottomLeftPoint) at ($(BaseStart)+(3,0)$); % weeeeell. it's bottom right, buuuut ... hehehehehehe
         \coordinate (BottomRightPoint) at (BaseStart);
         \coordinate (TopPoint) at ($(BaseStart)+(1.5,3.5)$);

         \foreach \y in {1,...,4}{
            \draw ($(BottomLeftPoint)!\y/6!(TopPoint)$) -- ($(BottomPoint)!\y/6!(TopPoint)$) -- ($(BottomRightPoint)!\y/6!(TopPoint)$);
         }

         % Top Dot:
         \filldraw[gray,draw=black,semithick] (TopPoint) circle (0.25);

         %\draw the Radio Waves
         \foreach \r in {0.5,1,...,2.5}{
            \draw[thin] (TopPoint) +(-35:\r cm) arc[radius=\r cm,delta angle=70,start angle=-35];
            \draw[thin] (TopPoint) +(180-35:\r cm) arc[radius=\r cm,delta angle=70,start angle=180-35];
         }
   \end{scope}
         % now, lets draw the rest :D


   % mobilenet1
   % \DrawLayers{mobilenet1.north}{physical,link,network}{mnet1-}{2}
   % \draw[btdl@color@background,line join=bevel] (mobilenet1.north) -- (mnet1-physical.south west) -- (mnet1-physical.south east) -- cycle;


   \node[server,scale=0.5] (mobserv1) at ($(TopPoint)+(-1.75,0.25)$) {};
   \node[server,scale=0.5] (mobserv2) at ($(TopPoint)+(-1.75,-0.75)$) {};

   \node  at ($(TopPoint)+(-0.75,-0.75)$) {\Large\faMobile};

   \node at ($(TopPoint)+(-1,0)$) {\Large\faMobile};

   % you will be the happy sender of the application-layer
   \node (sender-phone) at ($(TopPoint)+(-0.25,0.75)$) {\Large\faMobile};

   % \foreach \i in {3,2,1}{
   %     \DrawLayers{instrout\i.north}{physical,link,network}{instroute\i-}{2}
   %     \draw[btdl@color@background,line join=bevel] (instrout\i.north) -- (instroute\i-physical.south west) -- (instroute\i-physical.south east) -- cycle;
   % }

   \DrawLayers{mobserv1.north}{physical,link,network,transport,application}{source-}{-2}
   \DrawLayers{instserv1.north}{physical,link,network,transport,application}{receive1-}{2}

   \draw[link] (BottomLeftPoint) edge (mobilenet1);% (sender-phone) edge (TopPoint);

   \draw[link] (mobilenet1) edge (globisp2);

   \draw[fill=btdl@color@background,draw=black,line join=bevel] ([yshift=2mm]instserv1.north) -- (receive1-physical.south west) -- (receive1-physical.south east) -- cycle;

   \draw[fill=btdl@color@background,draw=black,line join=bevel] ([yshift=2mm]mobserv1.north) -- (source-physical.south west) -- (source-physical.south east) -- cycle;
   \node at ($(TopPoint)+(-0.25,-1.5)$) {\Large\faMobile};
   \node at ($(TopPoint)+(-1.5,-1.35)$) {\Large\faMobile};

   % \draw[very thick,green!75,Kite-Kite] (source-network.east) -- ++(0.55,0) -- (mnet1-network.north east) -- ++(4.25,0) -- ($(receive1-network.west)-(2,0)$) -- ++(2,0);

   % routing table for sicker router 123-42 - weil ich ihn mag, lass mich

   \node[below,align=center,scale=0.75,rectangle,draw,rounded corners] (routing-table) at ([xshift=2.5cm,yshift=-1.25cm]globisp4) {\begin{tabular}{lr}
         \toprule
            \scriptsize header val. & \scriptsize output link \\
         \midrule
            \(0100\) & 3 \\
            \(0101\) & 2 \\
            \(0111\) & 2 \\
            \(1011\) & 1 \\
            \(1111\) & 4 \\
         \bottomrule
   \end{tabular}};

   \draw[densely dotted,-Kite] (globisp4) -- (routing-table);
   \endscope
   % DIE EPISCHE LEGENDE VON QUATSCHOPOTTEL! IT IS THE REAL ONE!

   % \node[text width=5.25cm,scale=0.75] at ([xshift=4.75cm,yshift=1cm]globisp4.east) {%
   %       \begin{description}[leftmargin=*,widest={\tikz[baseline=-0.6ex]{\node[scale=0.145]{\router{}};}},align=right,itemsep=0.45em]
   %          \item[{\tikz[baseline=-0.6ex]{\node[scale=0.145]{\router{}};}}] Router, verwaltet die Komm\-unikation (primär Core)
   %          \item[{\tikz[baseline=-0.6ex]{\node[scale=0.75,server]{};}}] Server/PC, Hosts (primär Edge)
   %          \item[{\tikz[baseline=-0.6ex]{\node[scale=1.5]{\faMobile};}}] Mobiles Endgerät, mobiler Host (primär Edge)
   %          \item[{\tikz[baseline=-0.6ex]{\node[scale=0.75,rack switch]{};}}] (Wireless-)Link (primär Edge)
   %       \end{description}
   % };

\end{tikzpicture}
\end{uncoverenv}
\end{frame}
}