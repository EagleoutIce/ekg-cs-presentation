

\begin{frame}[c]{}
   \titlepage
\end{frame}

\begin{frame}{}
\begin{tikzpicture}[@O]
   \node[text width=.8\linewidth] at(current page.center) {
      \begin{itemize}
         \item Transistoren und Mikroelektronik
         \item Rekursion
         \item Quantenfeldtheorie
         \item Berechenbarkeits-Modelle % (Wo sind die Grenzen unseres Rechners sowohl physikalisch als auch aus der Informatik-Sicht)
         \item Rechnerarchitektur
         %  (mehr aus der elektrotechnischen/physikalischen Sicht aber wenn Interesse
         % besteht auch aus Software-Sicht)
         \item Verteilte Systeme
         \item Programmierparadigmen
         \item Algorithmen und Datenstrukturen
         \item Angewandte Gebiete der Informatik
         \item \ldots
      \end{itemize}
   };
\end{tikzpicture}
\end{frame}

\begin{frame}<1-2>
\begin{tikzpicture}[@O]
   \node at (current page.center) {%
      \only<1|handout:0>{\includegraphics[width=\paperwidth]{data/oct20-q41-gray.jpg}}%
      \only<2|handout:1>{\includegraphics[width=\paperwidth]{data/oct20-q41.jpg}}%
   };
\end{tikzpicture}
\end{frame}
