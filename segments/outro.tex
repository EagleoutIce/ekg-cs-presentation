

\begin{frame}[c]{}
   \titlepage
\end{frame}

\begin{frame}{}
\begin{tikzpicture}[@O]
   \node[text width=.8\linewidth] at(current page.center) {
      \begin{itemize}
         \item Transistoren und Mikroelektronik
         \item Rekursion
         \item Quantenfeldtheorie
         \item Berechenbarkeits-Modelle % (Wo sind die Grenzen unseres Rechners sowohl physikalisch als auch aus der Informatik-Sicht)
         \item Rechnerarchitektur
         %  (mehr aus der elektrotechnischen/physikalischen Sicht aber wenn Interesse
         % besteht auch aus Software-Sicht)
         \item Verteilte Systeme
         \item Programmierparadigmen
         \item Algorithmen und Datenstrukturen
         \item Angewandte Gebiete der Informatik
         \item \ldots
      \end{itemize}
   };
\end{tikzpicture}
\end{frame}

\begin{frame}<1-4>
\begin{tikzpicture}[@O]
   \node at (current page.center) {\clap{%
      \only<2|handout:1>{\includegraphics[height=\paperheight]{data/university-mix.jpg}}%
      \only<3|handout:2>{\includegraphics[height=\paperheight]{data/oxford-front.jpg}}%
      \only<4|handout:3>{\includegraphics[width=\paperwidth]{data/harvard-founders-library.jpg}}%
      \only<5|handout:0>{\includegraphics[width=\paperwidtdh]{data/oct20-q41-gray.jpg}}%
      \only<6|handout:4>{\includegraphics[width=\paperwidth]{data/oct20-q41.jpg}}%
   }};
   \only<3|handout:2>{\node[above right,T,text=btdl@color@background,scale=.5] at (current page.south west) {\href{https://www.cynic.org.uk/photos/Oxford/20110911/}{\textcopyright~Robin Stevens}~~\href{https://creativecommons.org/licenses/by-nc-nd/2.0/}{\ccbyncnd}};}
   \only<4|handout:3>{\node[above right,T,text=btdl@color@background,scale=.5] at (current page.south west) {\href{https://en.wikipedia.org/wiki/File:Howard_University_Founders_Library.jpg}{\textcopyright~Josh}~~\href{https://creativecommons.org/licenses/by/2.0/deed.en}{\ccby}};}
\end{tikzpicture}
\end{frame}
